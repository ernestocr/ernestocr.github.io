\documentclass[a4paper]{article}

\usepackage[margin=1.4in]{geometry}
\usepackage[utf8]{inputenc}
\usepackage[T1]{fontenc}
\usepackage{textcomp}
\usepackage{amsmath}
\usepackage{amssymb}
\usepackage{amsthm}
\usepackage{mathtools}
\usepackage{bm}

%\usepackage{newtxtext,newtxmath}
%\usepackage{tgtermes}
%\usepackage[italic]{mathastext}
%\usepackage{mathptmx}
\usepackage{mathpazo}

%\usepackage{enumitem}
%\usepackage{hyperref}
%\usepackage{graphicx}
%\usepackage{quantikz2}

\DeclareMathOperator{\N}{\mathbf{N}}
\DeclareMathOperator{\Z}{\mathbf{Z}}
\DeclareMathOperator{\Q}{\mathbf{Q}}
\DeclareMathOperator{\R}{\mathbf{R}}
\DeclareMathOperator{\C}{\mathbf{C}}
\DeclareMathOperator{\F}{\mathbf{F}}

%\DeclarePairedDelimiter\bra{\langle}{\rvert}
%\DeclarePairedDelimiter\ket{\lvert}{\rangle}
%\DeclarePairedDelimiterX\braket[2]{\langle}{
%\rangle}{#1\,\delimsize\vert\,\mathopen{}#2}

\theoremstyle{definition}
\newtheorem{defn}{Definition}
\newtheorem*{sol}{Solution}
%\theoremstyle{plain}
%\newtheorem{thm}{Theorem}
%\newtheorem{lem}{Lemma}
%\newtheorem{cor}{Corollary}
%\newtheorem{exa}{Example}
%\newtheorem{exeinner}{Exercise}
%\newenvironment{exe}[1]{%
%    \IfBlankTF{#1}
%    {\renewcommand{theexeinner}{\unskip}}
%    {\renewcommand\theexeinner{#1}}%
%    \exeinner
%}{\endexeinner}

\title{Solutions to \protect\\\textit{PROBLEMS IN QUANTUM
    MECHANICS} \protect\\
by I.I. Gol'dman and V.D. Krivchenkov}
\author{Ernesto Camacho}
\begin{document}
    \maketitle

    \begin{enumerate}
        \item Determine the energy levels and normalized
            wave functions of a particle in a potential box.
            The potential energy of the particle is $V =
            \infty$ for $x < 0$ and $x > a$, and $V = 0$ for
            $0 < x < a$.

            \begin{sol}
                We are tasked with solving the time
                independent Schrödinger equation $\hat{H}
                \psi(x) = E \psi(x)$ for the given
                potential. As the solutions must be zero
                outside the box, we have
                \begin{equation}
                    -\frac{\hbar^2}{2m}
                    \frac{d^2\psi(x)}{dx^2}
                    = E \psi(x)
                    \quad
                    \text{for } 0 \leq x \leq a.
                \end{equation}
                If $k^2 = 2mE \hbar^{-2}$ then the
                equivalent equation
                \begin{equation}
                    \frac{d^2\psi(x)}{dx^2} + k^2 \psi(x)
                    = 0,
                \end{equation}
                is a second order ordinary differential
                equation whose general solution is
                \begin{equation}
                    \psi(x)
                    = A \sin(k x) + B \cos(k x).
                \end{equation}
                The boundary condition at the origin implies
                that $\psi(0) = B = 0$, while the boundary
                condition at $a$ implies
                \begin{equation}
                    A \sin(k a)
                    = 0
                    \implies
                    k
                    = \frac{n\pi}{a}, \quad n \in \N_{0}.
                \end{equation}
                The trivial solution is discarded and
                recalling the definition of $k$ we obtain
                the allowed energy levels:
                \begin{equation}
                    %\frac{2mE}{\hbar^2} = n^2\pi^2 \implies
                    E_n
                    = \frac{\hbar^2 \pi^2 n^2}{2ma^2},
                    \quad n \in \N.
                \end{equation}
                The probabilistic interpretation of quantum
                mechanics requires that the valid wave
                functions be normalized, so
                \begin{align}
                    \int_{\R}
                    |\psi_n(x)|^2 \, dx
                    &= \int_{0}^{a} |A|^2 \sin^2\left(
                    \frac{n\pi}{a} x \right)  \, dx
                    \\
                    &= |A|^2 \int_0^{a} \frac{1 - \cos\left(
                    \frac{2n\pi}{a} x \right) }{2} \, dx \\
                    %&= \frac{|A|}{2} \left( a - \int_0^{2n\pi}
                    %\cos(u) \frac{a}{2n\pi} \, du\right) \\
                    %&= \frac{|A|}{2} \left( 
                    %    a + \frac{a}{2n\pi} \sin(2n\pi)
                    %\right) \\
                    &= |A|^2 \frac{a}{2}.
                \end{align}
                Therefore $A = \sqrt{2 / a}$ and we are
                done:
                \begin{equation}
                    E_n
                    = \frac{\hbar^2\pi^2}{2ma^2} n^2,
                    \quad
                    \psi_n(x)
                    = \sqrt{\frac{2}{a}} \sin\left(
                    \frac{\pi n}{a} x \right)
                    \quad
                    \text{for}
                    \quad
                    n \in \N.
                \end{equation}
            \end{sol}

            \break

        \item Show that a particle in a potential box (see
            the preceding problem) satisfies the relation
            \begin{equation}
                \overline{x}
                = \frac{1}{2} a,
                \quad 
                \overline{(x-\overline{x})^2}
                = \frac{a^2}{12} \left( 1 -
                \frac{6}{n^2\pi^2} \right).
            \end{equation}
            Show that for large values of $n$ the above
            result coincides with the corresponding
            classical solution.

            \begin{sol}
                The problem is vaguely expressed but the
                relation to be verified clarifies that we
                are to calculate the average position for an
                \textit{eigenfunction} of the given
                Hamiltonian. For any $\psi_n$ we have
                \begin{align}
                    \overline{x}
                    &= \frac{2}{a} \int_{0}^{a} x 
                    \sin^2\left( \frac{n\pi}{a}x \right) \,
                    dx \\
                    &= \frac{1}{a} \left[
                        \int_0^{a} x \, dx
                        - \int_0^{a} x\cos\left(
                        \frac{2n\pi}{a} x \right) \, dx
                    \right] \\
                    &= \frac{1}{a} \left[
                        \frac{a^2}{2} - 0
                    \right] \\
                    &= \frac{a}{2}.
                \end{align}
                Similarly,
                \begin{align}
                    \overline{(x-\overline{x})^2}
                    &= \int_{0}^a \frac{2}{a}
                    \left( x - \frac{a}{2} \right)^2
                    \sin^2\left( \frac{n\pi}{a} x \right) \,
                    dx \\
                    &= \frac{2}{a}
                    \int_0^{a} \left( x^2 - ax +
                    \frac{a^2}{4} \right) \sin^2 \left(
                    \frac{n\pi}{a} x \right) \, dx \\
                    &= \frac{2}{a} \left[
                        \int_0^{a} x^2 \sin^2\left(
                        \frac{n\pi}{a} x \right) \, dx
                        -a \int_0^{a} x \sin^2\left(
                        \frac{n\pi}{a}x \right) \, dx
                        + \frac{a^2}{4} \int_0^{a}
                        \sin^2\left( \frac{n\pi}{a}x \right)
                        \, dx
                    \right] \\
                    &= \frac{2}{a} \left[
                        \int_0^{a} x^2\sin^2\left(
                        \frac{n\pi}{a} x \right) 
                        - \frac{a^3}{4}
                        + \frac{a^3}{8}
                    \right] \\
                    %&= \frac{2}{a} \left[
                    %    \int_0^{a} x^2\sin^2\left(
                    %    \frac{n\pi}{a} x \right) 
                    %    - \frac{a^3}{8}
                    %\right] \\
                    &= \frac{2}{a} \left[
                        \frac{4 (\pi n a)^3 - 6 \pi
                        n a^3}{24 (\pi n)^3}
                        - \frac{a^3}{8}
                    \right] \\
                    &= a^2 \left( 
                        \frac{1}{3} - \frac{1}{4} -
                        \frac{1}{2 (\pi n)^2}
                    \right) \\
                    %&= a^2 \left( \frac{1}{12} - \frac{6}{12
                    %\pi^2n^2} \right) \\
                    &= \frac{a^2}{12} \left( 1 -
                    \frac{6}{\pi^2n^2} \right).
                \end{align}
                As $n \to \infty$ we see that
                \begin{equation}
                    \overline{x}
                    \to \frac{a}{2}
                    \quad\text{and}\quad
                    \overline{(x-\overline{x})^2}
                    \to \frac{a^2}{12}.
                \end{equation}
                The internet says that this matches with the
                classical solution, so we will believe it.
            \end{sol}

            \break

        \item Find the probability distribution for
            different values of the momentum of a particle
            in a potential box in the $n$th energy state.

            \begin{sol}
                The probability density for different
                values of the momentum is given by
                $|\hat{\psi}(p)|^2$ where
                \begin{align}
                    \hat{\psi}_n(p)
                    &= (\mathcal{F} \psi_n)(p) \\
                    &= \frac{1}{\sqrt{2\pi \hbar}}\int_{\R}
                    \psi_n(x) e^{-i px / \hbar} \, dx
                    \\
                    &= \frac{1}{\sqrt{a\pi\hbar}}
                    \int_{0}^{a} \sin\left( \frac{n\pi}{a} x
                    \right) e^{-ipx / \hbar} \, dx.
                \end{align}
                The integral is trivial but involved, so
                it's worthwhile doing once. Let $k = n\pi /
                a$, then
                \begin{align}
                    \int_0^{a} \sin(kx) e^{-ipx / \hbar} \,
                    dx
                    &= \int_0^{a} \left[
                        \frac{e^{ikx} - e^{-ikx}}{2i}
                    \right] e^{-ipx / \hbar} \, dx \\
                    &= \frac{1}{2i} \int_0^{a}
                    \left[ e^{ikx - ipx
                    / \hbar} - e^{-ikx - ipx / \hbar}
                    \right] \, dx \\
                    &= \frac{1}{2i} \int_0^{a} \left[
                        e^{-ix(p / \hbar-k)}
                        + e^{-ix(p / \hbar+k)}
                    \right] \, dx.
                \end{align}
                The first integral reads 
                \begin{align}
                    \int_0^{a} e^{-ix(p / \hbar - k)} \, dx
                    %&= \frac{i}{p/\hbar - k} \int_0^{-i(p /
                    %\hbar - k)a} e^{u} \, du \\
                    &= -\frac{1}{i(p/\hbar - k)} \left(
                    e^{-i(p / \hbar - k)a} - 1 \right),
                \end{align}
                while the second is
                \begin{align}
                    \int_0^{a} e^{-ix(p / \hbar + k)} \, dx
                    %&= \frac{i}{p/\hbar - k} \int_0^{-i(p /
                    %\hbar - k)a} e^{u} \, du \\
                    &= -\frac{1}{i(p/\hbar + k)} \left(
                    e^{-i(p / \hbar + k)a} - 1 \right).
                \end{align}
                Therefore
                \begin{align}
                    \hat{\psi}_n(p)
                    &= \frac{1}{\sqrt{a\pi\hbar}}
                    \left[
                        \frac{1}{2(p / \hbar - k)} \left( 
                            e^{-i(p / \hbar - k)a} - 1  
                        \right) + \frac{1}{2(p/\hbar + k)}
                        \left( e^{-i(p / \hbar + k)a} - 1
                        \right) 
                    \right] \\
                    &= \frac{1}{\sqrt{a\pi\hbar}} \left[
                        \frac{(p / \hbar + k) \left( e^{-i(
                        p / \hbar - k)a} - 1 \right) + (p /
                    \hbar - k) \left( e^{-i(p / \hbar + k)
                a} - 1 \right)}{2((p / \hbar)^2 - k^2)}
                    \right] \\
                    &= \frac{1}{\sqrt{a\pi\hbar}} \left[
                        (p / \hbar) \left( . \right)  - 2p /
                        \hbar
                    \right]
                \end{align}
                ...
            \end{sol}
            
            \break

        \item Find the energy levels and the wave functions
            of a particle in a non-symmetric potential well.
            Investigate the case $V_1 = V_2$.
    \end{enumerate}
\end{document}

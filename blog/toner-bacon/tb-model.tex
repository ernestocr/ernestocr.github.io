\documentclass[a4paper]{article}

\usepackage[margin=1.5in]{geometry}
\usepackage[utf8]{inputenc}
\usepackage[T1]{fontenc}
\usepackage{textcomp}
\usepackage{amsmath}
\usepackage{amssymb}
\usepackage{amsthm}
\usepackage{mathtools}
\usepackage{bm}

%\usepackage{newtxtext,newtxmath}
%\usepackage{tgtermes}
%\usepackage[italic]{mathastext}

%\usepackage{enumitem}
%\usepackage{hyperref}
%\usepackage{graphicx}
%\usepackage{quantikz2}

\DeclareMathOperator{\N}{\mathbf{N}}
\DeclareMathOperator{\Z}{\mathbf{Z}}
\DeclareMathOperator{\Q}{\mathbf{Q}}
\DeclareMathOperator{\R}{\mathbf{R}}
\DeclareMathOperator{\C}{\mathbf{C}}
\DeclareMathOperator{\F}{\mathbf{F}}

%\DeclarePairedDelimiter\bra{\langle}{\rvert}
%\DeclarePairedDelimiter\ket{\lvert}{\rangle}
%\DeclarePairedDelimiterX\braket[2]{\langle}{
%\rangle}{#1\,\delimsize\vert\,\mathopen{}#2}

%\theoremstyle{definition}
%\newtheorem{defn}{Definition}
%\newtheorem*{sol}{Solution}
%\theoremstyle{plain}
%\newtheorem{thm}{Theorem}
%\newtheorem{lem}{Lemma}
%\newtheorem{cor}{Corollary}
%\newtheorem{exa}{Example}
%\newtheorem{exeinner}{Exercise}
%\newenvironment{exe}[1]{%
%    \IfBlankTF{#1}
%    {\renewcommand{theexeinner}{\unskip}}
%    {\renewcommand\theexeinner{#1}}%
%    \exeinner
%}{\endexeinner}

\title{The Toner-Bacon Model}
\author{Ernesto Camacho}
\begin{document}
    \maketitle

    \section{The Toner-Bacon model}

    Given the state $\Psi$, Alice and Bob perform two
    projective measurements. Bob's measurment projects the
    quantum state into one of two mutually orthogonal
    vectors, which we denote by their corresponding Bloch
    vectors $\vec b_1$ and $\vec b_2 = -\vec b_1$. Alice
    performs a projective measurement on the states $\vec
    a_1$ and $\vec a_2 = -\vec a_1$. The joint probability
    of having outcomes $\vec a_\alpha$ and $\vec b_\beta$ is
    \begin{equation}
        \text{Pr}(\alpha,\beta)
        = \frac{1}{4}(1 - \vec a_\alpha \cdot \vec b_\beta).
    \end{equation}
    {\color{blue}
    To derive this expression, recall that the projection
    operators onto the states with Bloch vectors $\vec
    a_\alpha$ are given by
    \begin{equation}
        P_{\vec a_\alpha}
        = \frac{1}{2} \left( I + \vec a_\alpha \cdot \vec
        \sigma \right), 
    \end{equation}
    where $\vec \sigma$ is the ``vector'' of Pauli operators
    $X,Y,Z$. The same goes for the projection operator onto
    the states associated to $\vec b_{\beta}$. Therefore,
    the probability of observing outcomes $\alpha$ and
    $\beta$ is given by the Born rule
    \begin{equation}
        \text{Pr}(\alpha,\beta)
        %= \Tr\left( P_{\Psi} P_{\vec a_\alpha} \otimes
        %P_{\vec b_\beta} \right)
        = \langle \Psi, (P_{\vec a_\alpha} \otimes P_{\vec
        b_\beta}) \Psi \rangle.
    \end{equation}
    Since the singlet state $\Psi$ is invariant under any
    unitary operation, one can show that 
    \begin{equation}
        \langle \Psi, (\sigma_i \otimes I) \Psi \rangle
        = \langle \Psi, (I \otimes \sigma_i) \Psi \rangle
        = 0
        \quad\text{and}\quad
        \langle \Psi, (\sigma_i \otimes \sigma_j) \Psi
        \rangle
        = -\delta_{ij}.
    \end{equation}
    Therefore
    \begin{align}
        \text{Pr}(\alpha,\beta)
        &= \langle \Psi, (P_{\vec a_\alpha} \otimes P_{\vec
        b_\beta}) \Psi \rangle \\
        &= \left\langle \Psi, \frac{1}{2} \left( I +
        \vec a_\alpha \cdot \vec \sigma \right) 
        \otimes \frac{1}{2} \left( I + \vec b_\beta \cdot
        \vec \sigma \right) 
        \Psi \right\rangle \\
        %&= \frac{1}{4} \left\langle
        %    \Psi, \left( I \otimes I + \vec a_\alpha \cdot
        %    \vec \sigma \otimes I  + I \otimes \vec
        %    b_\beta \cdot \vec \sigma + (\vec a_\alpha \cdot
        %\vec \sigma) \otimes (\vec b_\beta \cdot \vec
        %\sigma) \right) \Psi
        %\right\rangle \\
        %&= \frac{1}{4} \left\langle
        %    \Psi, \left( I \otimes I + \vec a_\alpha \cdot
        %    \vec \sigma \otimes I  + I \otimes \vec
        %    b_\beta \cdot \vec \sigma + \sum_{i,j}^{}
        %a_\alpha^i b_\beta^j \sigma^i \otimes \sigma^j \right)
        %\Psi
        %\right\rangle \\
        &= \frac{1}{4} \left[
            1 + \sum_{i,j}^{} a_\alpha^i b_\beta^j 
            \langle \Psi, (\sigma^i \otimes \sigma^j) \Psi
            \rangle
        \right] \\
        %&= \frac{1}{4} \left[
        %    1 + \sum_{i,j}^{} a^{i}_\alpha b^{j}_\beta
        %    (-\delta_{ij})
        %\right] \\
        &= \frac{1}{4} \left( 1 - \vec a_\alpha \cdot \vec
        b_\beta \right). 
    \end{align}
    }

    Using a classical protocol that uses one bit of
    classical communication, we can reproduce the statistics
    exactly. The protocol is as follows: Alice and Bob share
    two random unit vectors $\vec \lambda_1$ and $\vec
    \lambda_2$. They are uncorrelated and uniformly
    distributed on the unit sphere. The goal is to generate
    the outcomes $\alpha$ and $\beta$ using only one bit of
    communication. To do so, Bob generates the outcome
    $\beta$ such that the vector $\vec b_\beta$ is closest
    to  $\vec \lambda_1$, that is,
    \begin{equation}
        \sgn \vec b_\beta \cdot \vec \lambda_1
        > 0.
    \end{equation}
    Bob then sends one bit $n \in \{-1,1\}$ to Alice, where
    \begin{equation}
        \sgn \vec b_\beta \cdot \vec \lambda_2.
    \end{equation}
    Alice then generates the outcome $\alpha$ such that
    \begin{equation}
        \sgn\left( \vec a_\alpha \cdot (\vec \lambda_1 + n
        \vec \lambda_2) \right) < 0.
    \end{equation}

    \begin{figure}[ht]
        \centering

        %\begin{blochsphere}[radius=3cm, tilt=20, rotation=0, opacity=0.1]
        %    \drawBallGrid[style={opacity=0.1}]{30}{30};

        %    \drawStatePolar{l1}{30}{0}; % (pi/6, 0)
        %    \node[above, right] at (l1) {$\vec \lambda_1$};

        %    \drawStatePolar{l2}{90}{120}; % (5pi/9, 2pi/3)
        %    \node[above, left] at (l2) {$\vec \lambda_2$};

        %    \drawStatePolar[statecolor=blue]{a1}{0}{0}; % (0, 0)
        %    \drawStatePolar[statecolor=blue]{a2}{180}{0}; % (pi, 0)
        %    \node[above, right] at (a1) {$\vec a_1$}; 
        %    \node[below, left] at (a2) {$\vec a_2$};

        %    \drawStatePolar[statecolor=red]{b1}{30}{180}; % (pi/6, pi)
        %    \drawStatePolar[statecolor=red]{b2}{150}{0}; % (5pi/6, 0)
        %    \node[above, left] at (b1) {$\vec b_1$};
        %    \node[below, right] at (b2) {$\vec b_2$};
        %\end{blochsphere}
        \caption{
            Here we have the two random vectors $\vec
            \lambda_1, \vec \lambda_2$ which are the hidden
            variables for a particular run of the experiment
            and the measurement directions of both
            {\color{blue} Alice} and {\color{red} Bob}.  For
            this setup, Bob outputs $\beta = 1$ since $\vec
            b_1 \cdot \vec \lambda_1 > 0$ and then sends
            1-bit of information, namely $n = \sgn \vec b_1
            \cdot \vec \lambda_2 = 1$. Finally, Alice
            generates the outcome $\alpha = 1$ since $\vec
            a_1 \cdot (\vec \lambda_1 + \vec \lambda_2) >
            0$.
        }
        \label{fig:toner-bacon-example}
    \end{figure}

    This protocol reproduces the outcomes according to the
    quantum probability $P(\alpha,\beta)$ [cite]. An example
    run is given in Fig. \ref{fig:toner-bacon-example}. Thus
    the Toner-Bacon model reproduces Bell state correlations
    by communicating a single bit. Alberto Montina then
    presents a generalization of this model that allows him
    to generalize to higher dimensions. But before this, it
    is instructive to analyze an alternative model for the
    Bell state correlations.
\end{document}

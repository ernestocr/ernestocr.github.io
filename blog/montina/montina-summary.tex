\documentclass[12pt,draft]{article}

\usepackage[margin=1.3in]{geometry}
\usepackage[utf8]{inputenc}
\usepackage[T1]{fontenc}

\usepackage{textcomp}
\usepackage{amsmath}
\usepackage{amssymb}
\usepackage{amsthm}
\usepackage{mathtools}
\usepackage{bm}
\usepackage{xcolor}
\usepackage{setspace}

\usepackage{enumitem}
\usepackage{hyperref}
\usepackage{graphicx}
%\usepackage{quantikz2}

\usepackage{tikz}
\usepackage{blochsphere}
%\usetikzlibrary{external}
%\tikzexternalize

\DeclareMathOperator{\Tr}{Tr}
\DeclareMathOperator{\sgn}{sgn}
\DeclareMathOperator{\N}{\mathbf{N}}
\DeclareMathOperator{\Z}{\mathbf{Z}}
\DeclareMathOperator{\Q}{\mathbf{Q}}
\DeclareMathOperator{\R}{\mathbf{R}}
\DeclareMathOperator{\C}{\mathbf{C}}
\DeclareMathOperator{\F}{\mathbf{F}}

\theoremstyle{definition}
\newtheorem{defn}{Definition}
\newtheorem*{sol}{Solution}
\theoremstyle{plain}
\newtheorem{thm}{Theorem}
\newtheorem{lem}{Lemma}
\newtheorem{cor}{Corollary}
\newtheorem{exa}{Example}

\onehalfspacing

\title{The Montina Model}
\author{Ernesto Camacho}
\begin{document}
    \maketitle

    As part of my research I've been tasked with studying
    the Montina model and how it could possibly relate to
    quasi-probability representations and negativity as an
    identifier of true quantum phenomena. The main questions
    to solve, although not all entirely related to the
    Montina model, are the following
    \begin{enumerate}
        \item In what way is the reduction of the phase
            space of the 8-state model related to negativity
            of frame representations.
        \item What benefits does Montina achieve relative to
            the KS-model?
        \item Is the increased ontic space (relative to the
            KS-model) for ``more non-negativity''?
        \item In what sense is the 8-state model contextual?
        \item In what sense is the KS-model contextual?
        \item In what sense is the Montina model contextual?
        \item Look into the draft.
    \end{enumerate}

    Alberto Montina's main interest in his 2011 paper
    \textit{Approximate simulation of entanglement with a
    linear cost of communication} was the simulation of the
    nonlocal correlations of entangled bipartite systems and
    the cost of communication needed to realize the
    simulation. We are interested in his model as an
    ontological model of a single qubit.

    \section{The Kochen-Specker model}

    Montina considers prepare and measure scenarios for a
    single qubit and gives a reformulation of the
    Kochen-Specker (KS) model for the spin of a free
    electron. This is from their 1967 paper, where they
    describe a hidden variable model for the qubit.  It can
    be used as a classical model of a quantum channel where
    the communicated classical information is infinite and
    encoded in a three dimensional unit vector.  Montina's
    contribution in this sense, is to show that it can be
    made into a model with finite communication cost.

    The experiment that Montina set out to simulate is the
    following. Bob prepares a qubit in a quantum state
    represented by a Bloch vector $b$. He sends the
    qubit to Alice, who performs a projective measurement
    $M_A = \{a_1, a_2\}$ with outcome states $
    a_1$ and $a_2 \equiv -a_1$. 

    \textbf{The protocol.} The classical simulation for this
    experiment is a reformulation of the KS model, and is
    given as follows.  Given the Bloch vector $b$
    corresponding to some preparation, Bob generates a unit
    vector $x$ according to the probability distribution
    \begin{equation}
        \rho(x | b)
        = \frac{1}{\pi} \, b \cdot x \,
        \theta(b \cdot x),
        \label{eqn:ks-rho}
    \end{equation}
    where $\theta(z)$ is the Heaviside function. He then
    sends $x_1$ to Alice, which constitutes a transmission
    of infinite information. Given the vector pair $M_A =
    \{a_1, a_2\}$, Alice generates the outcome $a_\alpha$
    according to the conditional probability
    \begin{equation}
        P(\alpha | x; M_A)
        = \theta(a_\alpha \cdot x)
        = \begin{cases}
            1 & \text{if } a_\alpha \cdot x > 0
            \\
            0 & \text{otherwise.}
        \end{cases}
    \end{equation}
    This model gives the correct quantum probability for the
    outcomes:
    \begin{equation}
        \text{Pr}(\alpha | b)
        = \frac{1}{2} \left( 1 + a_\alpha \cdot b
        \right) 
        = \int P(\alpha | x ; M_A) \rho(x, b) \, d x.
    \end{equation}

    {
        \color{blue} Let's reformulate Montina's formulation
        of the KS-model in terms of ontological
        representations of an operational theory. We require
        two maps
        \begin{equation}
            \mu
            : \mathcal{S} \to \mathcal{P}(S^2)
            \quad\text{and}\quad
            \xi
            : \mathcal{M} \to \mathcal{F}(S^2, \R),
        \end{equation}
        where $\mathcal{S}$ is the space of quantum states,
        in this case all qubit pure states, which can be
        viewed as $S^2$; and $\mathcal{M}$ is the set of
        orthogonal projections, which again, coincides with
        $S^2$. With these definitions, $\mu(\cdot | b)$ is a
        probability measure over the ontic state space $S^2$
        and $\xi(\alpha | \cdot)$ is a response function, or
        a value assignment corresponding to a projective
        measurement.

        To a given preparation $b \in S^2$, the
        Montina/KS-model assigns the following probability
        measure: 
        \begin{equation}
            \mu(x | b)
            = \frac{1}{\pi} \, b \cdot x \, \theta(b \cdot
            x)
            \quad \text{for all } x \in S^2.
        \end{equation}
        Then, given an ontic state $x$, for any orthogonal
        measurement $M = \{a_\alpha : \alpha = 1,2 \}$
        we have
        \begin{equation}
            \xi(\alpha | x) 
            = \theta(a_\alpha \cdot x).
        \end{equation}
        Clearly $\sum_{\alpha}^{} \xi(\alpha | x) = 1$ for
        all $x \in S^2$. The notation and definitions still
        leave room for some ambigüity, and in particular it
        would be more appropriate to label the response
        functions by the \textit{pair} of antipodal points,
        i.e., the measurement context. But, obviously in a
        two-dimensional Hilbert space, there is no room for
        context as the choice of a projection operator
        automatically picks out the PVM.

        To verify that this model truely is an ontological
        model, we first show that $\mu$ really does map
        states to probability measures. Let $H_b := \{x \in
        S^2 : b \cdot x > 0\}$, i.e., the hemisphere whose
        great circle is perpendicular to the vector $b$.
        Then
        \begin{align}
            \int_{S^2} \mu(x | b) \, d\Omega(x)
            &= \int_{S^2} \frac{1}{\pi} \, b \cdot x \,
            \theta(b \cdot x) \, d\Omega(x) \\
            &= \frac{1}{\pi} \int_{H_b} b \cdot x \,
            d\Omega(x) \\
            &= \frac{1}{\pi}
            \int_0^{2\pi} \int_{0}^{\pi / 2}
            \cos(\theta) \sin(\theta) \, d\theta \, d\phi \\
            &= 2 \int_0^{\pi / 2} \cos(\theta) \sin(\theta)
            \, d\theta \\
            &= 1.
        \end{align}
        If $P_a$ corresponds to the projection operator onto
        the state $a$, then to satisfy the Born rule, we
        must show that
        \begin{equation}
            \text{Pr}(\alpha | b)
            = \Tr(P_{a_\alpha} P_b)
            = \frac{1}{2} \left(
                1 + a_\alpha \cdot b
            \right), 
        \end{equation}
        is equal to
        \begin{equation}
            \int_{S^2} \xi(\alpha | x) \mu(x | b) \,
            d\Omega(x).
        \end{equation}
        First observe that
        \begin{align}
            \int_{S^2} \xi(\alpha | x) \mu(x | b) \,
            d\Omega(x)
            &= \int_{S^2} \theta(a_\alpha \cdot x) \left[
                \frac{1}{\pi} \, b \cdot x \, \theta(b \cdot
                x) 
            \right] \, d\Omega(x) \\
            &= \frac{1}{\pi} \int_{H_{a_\alpha} \cap H_b}
            b \cdot x \, d\Omega(x).
        \end{align}
        So we have a integral over a spherical lune defined
        by two hemispheres. To simplify this somewhat
        non-trivial integral we must choose coordinates
        wisely.  Geometrically we choose a polar axis
        perpendicular to both the preparation $b$ and
        measurement $a$, so that the integration region is
        defined as a range of allowed azimuthal angles. Let
        $\alpha$ be the angle between $b$ and $a$ and choose
        coordinates such that $b = (0,1,0)$ and $a =
        (-\sin\alpha, \cos\alpha,0)$. If $x =
        (\sin\theta\cos\phi,\sin\theta\sin\phi,\cos\theta)$,
        then $x \in H_b \cap H_b$ if
        \begin{equation}
            b \cdot x
            = \sin\theta\sin\phi > 0,
        \end{equation}
        which implies that $\phi \in (0,\pi)$ (since
        $\sin\theta > 0$ for $\theta \in (0,\pi)$).
        Furthermore, $x$ must satisfy
        \begin{align}
            a \cdot x
            &= -\sin\alpha \sin\theta\cos\phi + \cos\alpha
            \sin\theta\sin\phi \\
            &= \sin\theta \left( 
                -\sin\alpha\cos\phi + \cos\alpha\sin\phi
            \right) \\
            &= \sin\theta \sin(\phi - \alpha) \\
            &> 0,
        \end{align}
        implying that $\phi - \alpha \in (0,\pi)$, i.e.,
        $\phi \in (\alpha,\alpha + \pi)$. Satisfying both
        conditions means that $\phi \in (\alpha, \pi)$.
        Therefore
        \begin{align}
            \int_{H_b \cap H_a} b \cdot x \, d\Omega(x)
            &= \int_{0}^{\pi} \int_{\alpha}^{\pi} \sin\phi
            \sin^2\theta \, d\phi \, d\theta \\
            &= \frac{\pi}{2}
            \int_{\alpha}^{\pi} \sin\phi \, d\phi
            \\
            &= \frac{\pi}{2} \left(
                -\cos(\pi) + \cos(\alpha)
            \right) \\
            &= \frac{\pi}{2} \left( 
                1 + \cos(\alpha)
            \right). 
        \end{align}
        Since $\cos(\alpha) = b \cdot x$ the calculation is
        complete.
        
        This model gives a deterministic explanation for
        spin measurements on a free electron. It does not
        deal with dynamics. Furthermore, a natural question
        arises, is the model measurement non-contextual? At
        least for projection value measures the question
        does not even arise, since the context is fully
        determined once a choice of projector is made.
    }

    So Montina has simply reformulated the KS-model for a
    qubit, which deals with pure states, spin measurements
    and is non-contextual. For his communication scenario,
    the model involves the transmission of an infinite
    amount of information, i.e., the unit vector $x$ which
    is drawn from a distribution that is \textit{dependent}
    on the preparation $b$. To show how this can be done
    with only two bits of classical information, Montina
    first modifies the KS-model to include an additional
    hidden variable in the form of another unit vector.

    Define the vectors $y_1 := x_1 + x_2$ and $y_2 := x_1 -
    x_2$.  The probability distribution $\rho$ in Eq.
    \eqref{eqn:ks-rho} can be obtained from the following
    \textit{uniform} distribution
    \begin{equation}
        \rho(x_1, x_2 | b)
        = \frac{1}{4\pi^2} \theta(b \cdot y_1)
        \theta(b \cdot y_2)
    \end{equation}
    by integrating out the unit vector $x_2$. 
    {
        \color{blue}
        So we now have a new map from states to probability
        measures over the ontic space $S^2 \times S^2$
        defined as the uniform distribution
        \begin{equation}
            \mu(x_1, x_2 | b)
            = \frac{1}{4\pi^2}
            \theta(b \cdot y_1) \theta(b \cdot y_2),
        \end{equation}
        where $y_1 = x_1 + x_2$ and $y_2 = x_1 - x_2$.
        Marginalization over $x_2$ should recover
        $\mu(\cdot | b)$. First note that the $\mu$ is
        non-zero only when
        \begin{equation}
            b \cdot (x_1 + x_2) > 0
            \quad\text{and}\quad
            b \cdot (x_1 - x_2) > 0,
        \end{equation}
        equivalently
        \begin{equation}
            -b \cdot x_1 < b \cdot x_2 < b \cdot x_1,
        \end{equation}
        for fixed vectors $b$ and $x_1$. Suppose $b =
        (0,0,1)$, and let $\theta_1$ be the angle between
        vectors $b$ and $x_1$, therefore the inequality
        above can be expressed as $-\cos\theta_1 <
        \cos\theta < \cos\theta_1$, which implies that
        $\cos^{-1}(b \cdot x_1) < \theta <
        \cos^{-1}(-b \cdot x_1)$. And so the integral is
        \begin{align}
            \int_{S^2} \mu(x_1,x_2|b) d\Omega(x_2) 
            &= \frac{1}{4\pi^2} \int_{S^2} 
            \theta(b \cdot (x_1 + x_2)) \theta(b \cdot (x_1
            - x_2)) \, d\Omega(x_2) \\
            %&= \frac{1}{4\pi^2}
            %\int_{\cos^{-1}(b \cdot x_1)}^{\cos^{-1}(-b
            %\cdot x_1)}
            %2\pi \sin(\theta) \, d\theta \\
            &= \frac{1}{4\pi^2}
            \int_{b \cdot x_1}^{-b \cdot x_1}
            2\pi \sin(\theta) \, d\theta \\
            %&= \frac{1}{2\pi} \left[
            %    \cos(\theta)
            %\right]_{-\cos^{-1}(\theta_1)}^{\cos^{-1}(\theta_1)}
            %\\
            &= \frac{1}{2\pi} \left( 
                -(-b \cdot x_1) - -(b \cdot x_1)
            \right) \\
            &= \frac{1}{\pi} \, b \cdot x_1.
        \end{align}
        To define the marginalized function for all of
        $S^2$, we add on the Heaviside function to obtain
        the original distribution
        \begin{equation}
            \mu(x_1 | b)
            = \frac{1}{\pi} \, b \cdot x_1 \, \theta(b \cdot
            x_1)
            \quad
            \text{for all } x_1 \in S^2.
        \end{equation}
        The ontological model now reproduces the Born rule
        by averaging the response function $\xi(\alpha|x_1)$ 
        by the probability function $\rho(x_1,x_2|b)$ over
        the ontic space $S^2 \times S^2$.
    }

    The ontic space is now sampled from a uniform
    distribution over subsets defined by the preparation
    procedure $b$. For Montina's considerations, this still
    constitutes and infinite amount of information. We can
    get another exact simulation of a qubit by introducing
    two more hidden variables $n_1$ and $n_2$ which are one
    bit each, drawn from $\{-1,1\}$. As we will see, this
    allows for the sampling of the two unit vectors to be
    \textit{independent} of $b$, which for communication
    purposes reduces the transmission cost. Montina defines
    a probability distribution of the two unit vectors
    conditioned on the bits, 
    \begin{equation}
        \rho(x_1, x_2 | n_1, n_2; b)
        = \frac{1}{4\pi^2} \, \theta(n_1 b \cdot 
        y_1) \, \theta(n_2 b \cdot y_2),
    \end{equation}
    and a response function
    \begin{equation}
        P(\alpha | x_1, x_2, n_1, n_2; M_A)
        = \theta\left[a_\alpha \cdot (n_1 y_1 +
        n_2 y_2)\right],
    \end{equation}
    for any choice $n_i = \pm 1$. Any particular choice
    defines an ontological model which can be
    derived from the original two vector model above. For
    example with $n_1 = 1$ and $n_2 = -1$, these equations
    can be derived from the original distribution by
    switching $x_1$ and $x_2$. With $n_1 = -1$ and $n_2 =
    -1$, we have to flip the direction of $x_1$ and $x_2$.
    Finally, with $n_1= -1$ and $n_2 = 1$, we have to
    exchange and flip the vectors.
    {
        \color{red}
        Let $n_1 = 1$ and $n_2 = -1$. Then the Montina model
        defines
        \begin{align}
            \rho(x_1, x_2, | 1, -1, b)
            &= \frac{1}{4\pi^2} \, 
            \theta(b \cdot y_1) \, \theta(-b \cdot y_2) \\
            &= \frac{1}{4\pi^2} \,
            \theta(b \cdot (x_2 + x_1)) \, 
            \theta(b \cdot (x_2 - x_1)),
        \end{align}
        and
        \begin{align}
            P(\alpha | x_1, x_2, 1, -1, M_A)
            &= \theta\left[
                a_{\alpha} \cdot (y_1 - y_2)
            \right] \\
            &= \theta\left[
                a_\alpha \cdot (2 x_2)
            \right] \\
            &= \theta(a_\alpha \cdot x_2).
        \end{align}
        As stated above, this is just the original model
        with the vectors $x_1$ and $x_2$ switched. The other
        options can be similarly derived. The important
        thing is that because of this all of these models
        simulate the qubit exactly.
    }

    Now consider the scenario in which the indices $n_i$ are
    randomly generated with probability $\rho_I(n_1,n_2) = 1
    / 4$. Then the \textit{joint} probability distribution
    of $x_i$ and $n_i$ can be suitably written in the
    form
    \begin{equation}
        \rho(x_1, x_2, n_1, n_2 | b)
        = \rho(n_1, n_2, | x_1, x_2; b)
        \rho_v(x_1, x_2),
    \end{equation}
    where
    \begin{equation}
        \rho(n_1, n_2, | x_1, x_2; b)
        := \theta(n_1 b \cdot y_1) \theta(n_2 
        b \cdot y_2)
    \end{equation}
    and
    \begin{equation}
        \rho_v(x_1, x_2)
        = \frac{1}{(4\pi)^2}.
    \end{equation}
    This model also reproduces the correct statistics but
    also has the property that the marginal probability
    distribution $\rho_v(x_1, x_2)$ of the vectors
    $x_1$ and $x_2$ does not depend on the
    prepared state $b$. The dependence is only in the
    conditional distribution $\rho(n_1,n_2 | x_1, 
    x_2; b)$ of the discrete indices $n_i$.

    {
        \color{red}
        The joint probability measure $\rho(\cdot|b) : S^2
        \times S^2 \times \{-1,1\}^2 \to [0,1]$ can be
        expressed in two equivalent forms:
        \begin{equation}
            \rho(x_1,x_2|n_1,n_2,b)\rho_I(n_1,n_2),
            \quad\text{and}\quad
            \rho(n_1,n_2,|x_1,x_2,b)\rho_v(x_1,x_2)
        \end{equation}
        for uniform distributions $\rho_v$ and $\rho_I$.
        To see this, note that by definition of the
        distribution of the vectors conditioned on the bits,
        we have
        \begin{align}
            p(x_1,x_2|n_1,n_2,b) \rho_I(n_1,n_2)
            &= \left(
                \frac{1}{4\pi^2} 
            \theta(n_1 b \cdot y_1)\theta(n_2 b \cdot y_2) 
            \right) \left( \frac{1}{4} \right) \\
            &= \left( \theta(n_1 b \cdot y_1)\theta(n_2 b \cdot
            y_2)\right) \left( 
                \frac{1}{(4\pi)^2}
            \right) \\
            &= \rho(n_1,n_2|x_1,x_2,b) \rho_v(x_1,x_2),
        \end{align}
        so
        \begin{equation}
            \rho_v(x_1,x_2)
            = \frac{1}{(4\pi)^2}
            \quad\text{and}\quad
            \rho_n(n_1,n_2)
            = \frac{1}{4}.
        \end{equation}
        Essentially, the randomness has been shifted from
        the unit vectors to the bits, i.e., the joint
        distribution can be expressed as the product of
        distribution of the bits conditional on the unit
        vectors, which are uniformly distributed. The
        important advantage with this expression is the the
        unit vectors do not depend on $b$ anymore and can
        be seen as \textit{shared} randomness between the
        parties.
    }

    So this procedure is a protocol for simulating the
    communication of a qubit with just two bits of classical
    communication. {\color{blue}
        Bob and Alice share the random vectors $x_1$
        and $x_2$, generated according to the uniform
        probability distribution; then, given the
        preparation $b$, Bob generates the discrete indices
        $n_i$ according to the rule; he then sends them to
        Alice, whom, given a measurement $M_A$, generates
        the outcome $a_\alpha$ in a deterministic manner.
    } The protocol can be summarized as follows:
    \begin{enumerate}
        \item Alice and Bob hold the same random pair of
            unit vectors $(x_1, x_2) \in S^2
            \times S^2$.
        \item Bob prepares a quantum state according to
            $b$, and deterministically sets the two
            bits according to
            \begin{equation}
                n_1 = \theta(b \cdot y_1)
                \quad\text{and}\quad
                n_2 = \theta(b \cdot y_2),
            \end{equation}
            where $y_1 = x_1 + x_2$ and $
            y_2 = x_1 - x_2$. He then sends the
            two bits to Alice.
        \item Alice outputs the outcome $\alpha$ whose Bloch
            vector maximizes 
            \begin{equation}
                a_\alpha \cdot (n_1 y_1 + n_2 y_2).
            \end{equation}
    \end{enumerate}


    \section{Dynamics}

    \subsection{Measurement collapse}

    A feature of the statistical interpretation of quantum
    theory is that the collapse of the wave function induced
    by measurement is nothing more than an actualization of
    the probability distribution. In the KS-model, we see
    that after a measurement outcome, the probability
    measure corresponding to the state $b$ now becomes the
    measure corresponding to the observed outcome projector
    $a_\alpha$.

    \subsection{Unitary transformations}

    Unitary transformations in the KS model simply consist
    of applying the corresponding $O(3)$ rotation on the
    ontic state $x_1$, for a given $U \in \text{SU}(2)$
    acting on the preparation $b$. In the Montina model, the
    rotation must be applied to both unit vectors, but as we
    will see, the classical bits stay unaffected.

    Recall that a rotation $R_n(\theta)$ of angle $\theta$
    about the axis given by the vector $n$, corresponds to
    the unitaires $\pm U$ defined by
    \begin{equation}
        U
        = \exp\left( -\frac{i}{2} \theta \, n \cdot \sigma
        \right), 
    \end{equation}
    where $\sigma$ is the ``vector'' of Pauli matrices. If
    $U$ acts on the state $P_b$ via $U P_b U^{\dag}$, then
    the corresponding Bloch vector $b$ is transformed by
    $R_n(\theta)$. Since the dot product is invariant under
    rotations, the transformed probability measure
    $\mu(\cdot|b)$ leaves the statistics unaltered, and so
    state transformations simply induce a transformation on
    the ontic space. For the KS-model, if $R_U$ corresponds
    to $U$, then 
    \begin{align}
        \mu(x | R_U b)
        &= \frac{1}{\pi} (R_U b) \cdot x \, \theta(R_U b
        \cdot x) \\
        &= \frac{1}{\pi} (R_U b) \cdot (R_U R_U^{-1} x) \,
        \theta((R_U b) \cdot (R_U R_U^{-1}x)) \\
        &= \frac{1}{\pi} b \cdot (R_U^{-1}x) \,
        \theta(b \cdot R_U^{-1}x) \\
        &= \mu(R_U^{-1} x | b).
    \end{align}
    In other words, the KS-model is covariant under
    $\text{SU}(2)$. For Montina's model we have a similar
    situation, the unit vectors are rotated but the bits
    stay the same. For the joint distribution $\rho$ we see
    that
    \begin{align}
        \rho(x_1,x_2,n_1,n_2 | R_U b)
        &= \rho(n_1,n_2 | x_1,x_2; R_U b) \rho_v(x_1,x_2) \\
        &= \frac{1}{(4\pi)^2} 
        \theta(n_1 (R_U b) \cdot y_1) \theta(n_2 (R_U b)
        \cdot y_2) \\
        &= \frac{1}{(4\pi)^2} \theta(n_1 b \cdot
        (R_U^{-1}y_1)) \theta(n_2 b \cdot (R_U^{-1} y_2)) \\
        &= \rho(n_1,n_2 | R_U^{-1}x_1, R_U^{-1}x_2; R_U b)
        \rho_v(R_U^{-1}x_1, R_U^{-1}x_2) \\
        &= \rho(R_U^{-1}x_1, R_U^{-1}x_2, n_1, n_2 | R_U b).
    \end{align}
    So Montina's model is also covariant under
    $\text{SU}(2)$, and the corresponding action on the
    ontic space is just the $\text{SO}(3)$ rotation on both
    unit vectors. Therefore this model can deal with unitary
    channels acting on pure states. The question of whether
    it can deal with arbitrary quantum channels and mixed
    states becomes more subtle and essentially depends on
    whether the ontological is convex.

    \subsection{Convexity of the Montina model}

    \subsection{Process fidelity}

    In the glassboard Joseph wrote:
    \begin{equation}
        F(G, \tilde G)
        = \mathbf{E}_{g \in C_1}\left[
        \frac{1}{d^2} \Tr\left( \tilde G \cdot G^{-1}
        \right) \right]
    \end{equation}
    where $G := g \otimes \overline{g}$ for $g \in
    \text{SU}(2)$ and $\tilde G := \tilde g \otimes
    \overline{\tilde g}$.

\end{document}


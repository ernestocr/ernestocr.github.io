\documentclass[12pt,draft]{article}

\usepackage[margin=1.3in]{geometry}
\usepackage[utf8]{inputenc}
\usepackage[T1]{fontenc}

\usepackage{textcomp}
\usepackage{amsmath}
\usepackage{amssymb}
\usepackage{amsthm}
\usepackage{mathtools}
\usepackage{bm}
\usepackage{xcolor}
\usepackage{setspace}

\usepackage{enumitem}
\usepackage{hyperref}
\usepackage{graphicx}
%\usepackage{quantikz2}

\usepackage{tikz}
\usepackage{blochsphere}
%\usetikzlibrary{external}
%\tikzexternalize

\DeclareMathOperator{\Tr}{Tr}
\DeclareMathOperator{\sgn}{sgn}
\DeclareMathOperator{\N}{\mathbf{N}}
\DeclareMathOperator{\Z}{\mathbf{Z}}
\DeclareMathOperator{\Q}{\mathbf{Q}}
\DeclareMathOperator{\R}{\mathbf{R}}
\DeclareMathOperator{\C}{\mathbf{C}}
\DeclareMathOperator{\F}{\mathbf{F}}

\theoremstyle{definition}
\newtheorem{defn}{Definition}
\newtheorem*{sol}{Solution}
\theoremstyle{plain}
\newtheorem{thm}{Theorem}
\newtheorem{lem}{Lemma}
\newtheorem{cor}{Corollary}
\newtheorem{exa}{Example}

\onehalfspacing

\title{The Montina Model}
\author{Ernesto Camacho}
\begin{document}
    \maketitle

    As part of my research I've been tasked with studying
    the Montina model and how it could possibly relate to
    quasi-probability representations and negativity as an
    identifier of true quantum phenomena. The main questions
    to solve, although not all entirely related to the
    Montina model, are the following
    \begin{enumerate}
        \item In what way is the reduction of the phase
            space of the 8-state model related to negativity
            of frame representations.
        \item What benefits does Montina achieve relative to
            the KS-model?
        \item Is the increased ontic space (relative to the
            KS-model) for ``more non-negativity''?
        \item In what sense is the 8-state model contextual?
        \item In what sense is the KS-model contextual?
        \item In what sense is the Montina model contextual?
        \item Look into the draft.
    \end{enumerate}

    Alberto Montina's main interest in his 2011 paper
    \textit{Approximate simulation of entanglement with a
    linear cost of communication} was to investigate the
    simulation of entanglement. More precisely, simulation
    of the nonlocal correlations of entangled bipartite
    systems and the cost of communication needed to realize
    the simulation.

    Bell's theorem precludes the possibility of explaining
    the outcome probabilities by a local classical theory,
    therefore any exact simulation of this scenario needs
    some communication between the parties involved in the
    standard Alice and Bob scenario in which they share a
    system in the singlet state
    \begin{equation}
        \Psi
        = \frac{1}{\sqrt{2}}\left(
            \psi^{(1)}_{+}\psi^{(2)}_{-}
        - \psi^{(1)}_{-}\psi^{(2)}_{+}\right),
    \end{equation}
    where the $\psi^{(i)}_{\pm}$ are the $\pm 1$-eigenstates
    of the Pauli operator $Z$, which is proportional to the
    spin operator in the $z$ direction.

    The trivial solution is that Bob sends the exact
    information of his measurement choice, which is an
    infinite amount of information, but it was shown by
    multiple authors that a finite amount of communication
    is sufficient for exactly reproducing the outcome
    probabilities. Alberto Montina reviews one such model,
    given by Toner and Bacon [cite], and then later derives
    an alternative model which has a generalization to $d$ 
    dimensions.

    \section{The Kochen-Specker model}

    Following the presentation of the Toner-Bacon model for
    simulating entanglement, Montina considers prepare and
    measure scenarios for a single qubit and gives a
    reformulation of the Kochen-Specker (KS) model for the
    spin of a free electron. This is from their 1967 paper,
    where they describe a hidden variable model for the
    qubit and from this, one can obtain a classical model of
    a Bell state.  It can be seen as a classical model of a
    quantum channel where the communicated classical
    information is infinite and encoded in a three
    dimensional unit vector. Montina's contribution in this
    sense, is to show that it can be made into a model with
    finite communication cost.

    \subsection{Quantum channel}

    The experiment that Montina set out to simulate is the
    following. Bob prepares a qubit in a quantum state
    represented by a Bloch vector $\vec b$. He sends the
    qubit to Alice, who performs a projective measurement
    $M_A = \{\vec a_1, \vec a_2\}$ with outcome states $\vec
    a_1$ and $\vec a_2 \equiv -\vec a_1$. 

    \textbf{The protocol.} The classical simulation for this
    experiment is a reformulation of the KS model, and is
    given as follows.  Given the Bloch vector $\vec b$, Bob
    generates a unit vector $\vec x_1$ according to the
    probability distribution
    \begin{equation}
        \rho(\vec x_1 | \vec b)
        = \frac{1}{\pi} \, \vec b \cdot \vec x_1 \,
        \theta(\vec b \cdot \vec x_1),
    \end{equation}
    where $\theta(z)$ is the Heaviside function. He then
    sends $\vec x_1$ to Alice, which in general, constitutes
    a transmission of infinite information. Given the vector
    pair $M_A = \{\vec a_1, \vec a_2\}$, Alice generates the
    outcome $\vec a_\alpha$ according to the conditional
    probability
    \begin{equation}
        P(\alpha | \vec x_1; M_A)
        = \theta(\vec a_\alpha \cdot \vec x_1)
        = \begin{cases}
            1 & \text{if } \vec a_\alpha \cdot \vec x_1 > 0
            \\
            0 & \text{otherwise.}
        \end{cases}
    \end{equation}
    {\color{blue}
        Note that this is not really a conditional
        probability \textit{unless} the $\vec x_1$ is also
        fixed. I prefer to see this as a deterministic
        function of the hidden variable, i.e., Alice
        deterministically generates the vector closest to
    $\vec x_1$.} This model gives the correct quantum
    probability for the outcomes:
    \begin{equation}
        \text{Pr}(\alpha | \vec b)
        = \frac{1}{2} \left( 1 + \vec a_\alpha \cdot \vec b
        \right) 
        = \int P(\alpha | \vec x_1 ; M_A) \rho(\vec
        x_1, \vec b) \, d \vec x_1.
    \end{equation}
    {\color{blue}
        \begin{proof}
            It is clear that
            \begin{equation}
                \text{Pr}(\alpha | \vec b)
                = \frac{1}{2} \left( 
                    1 + \vec a_\alpha \cdot \vec b
                \right) 
            \end{equation}
            follows from the Born rule and the fact that any
            measurement basis projection (and state) can be
            decomposed as
            \begin{equation}
                P_{a_\alpha}
                = \frac{1}{2} \left( I + \vec a_\alpha \cdot
                \vec \sigma \right).
            \end{equation}
            To verify that the integral matches, first note
            that by definition of the Heaviside function we
            have
            \begin{align}
                \int P(\alpha | \vec x_1; M_A)
                \rho(\vec x_1, \vec b) \, d^2 \vec x_1 
                &= \frac{1}{\pi} \int \vec b \cdot \vec x_1 
                \theta(\vec a_\alpha \cdot \vec x_1)
                \theta(\vec b \cdot \vec x_1) \, d^2 \vec
                x_1 \\
                &=  \frac{1}{\pi} \int_{\vec a_\alpha \cdot
                \vec x_1 > 0 \cap \vec b \cdot \vec x_1 > 0}
                \vec b \cdot \vec x_1 \, d^2 \vec x_1.
            \end{align}
            So essentially we have an integral on the
            intersection of two hemispheres given by the two
            fixed vectors $\vec a_\alpha$ and $\vec b$.
            These integrals are a bit cumbersome, but using
            an appropriate choice of coordinates one can
            show that
            \begin{equation}
                \int_{\vec a_\alpha \cdot
                \vec x_1 > 0 \cap \vec b \cdot \vec x_1 > 0}
                \vec b \cdot \vec x_1 \, d^2 \vec x_1
                = \frac{\pi}{2}(1 + \vec a_\alpha \cdot \vec
                b),
            \end{equation}
            from which the result follows.
            %To simplify matters, let's choose a coordinate
            %system where  $\vec b$ has polar and azimuthal
            %angles equal to zero so that in spherical
            %coordinates we obtain
            %\begin{align}
            %    \frac{1}{\pi} \int_{\vec a_\alpha \cdot
            %    \vec x_1 > 0 \cap \vec b \cdot \vec x_1 > 0}
            %    \vec b \cdot \vec x_1 \, d^2 \vec x_1
            %    &= \frac{1}{\pi} \int_{\vec a_\alpha \cdot
            %    \vec x_1 > 0 \cap \cos(\theta) > 0}
            %    \cos(\theta) \, \sin(\theta) d\theta \,
            %    d\phi \\
            %    &= \ldots
            %\end{align}
        \end{proof}
    }

    {
        \color{red} Let's reformulate Montina's formulation
        of the KS-model in terms of ontological
        representations of an operational theory. We require
        two maps (we put no extra restrictions on these maps
        as most representations do, in particular, we do not
        require linearity):
        \begin{equation}
            \mu
            : \mathcal{S} \to \mathcal{P}(S^2)
            \quad\text{and}\quad
            \xi
            : S^2 \to \mathcal{F}(\mathcal{M}, \R),
        \end{equation}
        where $\mathcal{S}$ is the space of quantum states,
        in this case all qubit pure states, which can be
        viewed as $S^2$; and $\mathcal{M}$ is the set of
        orthogonal measurements, which can be seen as the
        disjoint union of pairs of antipodal points $M =
        \{a_1, a_2\} \subset S^2$ with $a_2 = -a_1$. With
        these definitions, $\mu(\cdot | b)$ is a probability
        measure over the ontic state space $S^2$ and
        $\xi(\cdot | x)$ is a response function, or a value
        assignment over projective measurements.

        To a given preparation $b \in S^2$, the
        Montina/KS-model assigns the following probability
        measure: 
        \begin{equation}
            \mu(x | b)
            = \frac{1}{\pi} \, b \cdot x \, \theta(b \cdot
            x)
            \quad \text{for all } x \in S^2.
        \end{equation}
        Then, given an ontic state $x$, for any orthogonal
        measurement $M = \{a_\alpha : \alpha = 1,2 \}$
        we have
        \begin{equation}
            \xi(\alpha | x) 
            = \theta(a_\alpha \cdot x).
        \end{equation}
        The notation and definitions still leave room for
        some ambigüity, but it does not cause much problem
        once the objects are actually understood. We wish to
        verify a few things, first we show that $\mu$ really
        does map states to probability measures. Let $H_b :=
        \{x \in S^2 : b \cdot x > 0\}$, i.e., the hemisphere
        defined by the vector $b$. Then
        \begin{align}
            \int_{S^2} \mu(x | b) \, d\Omega(x)
            &= \int_{S^2} \frac{1}{\pi} \, b \cdot x \,
            \theta(b \cdot x) \, d\Omega(x) \\
            &= \frac{1}{\pi} \int_{H_b} b \cdot x \,
            d\Omega(x) \\
            &= \frac{1}{\pi}
            \int_0^{2\pi} \int_{0}^{\pi / 2}
            \cos(\theta) \sin(\theta) \, d\theta \, d\phi \\
            &= 2 \int_0^{\pi / 2} \cos(\theta) \sin(\theta)
            \, d\theta \\
            &= 1.
        \end{align}
        If $P_a$ corresponds to the projection operator onto
        the state $a$, then to satisfy the Born rule, we
        must show that
        \begin{equation}
            \text{Pr}(\alpha | b)
            = \Tr(P_{a_\alpha} P_b)
            = \frac{1}{2} \left(
                1 + a_\alpha \cdot b
            \right), 
        \end{equation}
        is equal to
        \begin{equation}
            \int_{S^2} \xi(\alpha | x) \mu(x | b) \,
            d\Omega(x).
        \end{equation}
        First observe
        \begin{align}
            \int_{S^2} \xi(\alpha | x) \mu(x | b) \,
            d\Omega(x)
            &= \int_{S^2} \theta(a_\alpha \cdot x) \left[
                \frac{1}{\pi} \, b \cdot x \, \theta(b \cdot
                x) 
            \right] \, d\Omega(x) \\
            &= \frac{1}{\pi} \int_{H_{a_\alpha} \cap H_b}
            b \cdot x \, d\Omega(x).
        \end{align}
        So we have a integral over a spherical lune defined
        by two hemispheres. To simplify this rather
        non-trivial integral we choose coordinates wisely.
        Geometrically the main idea is to choose a polar
        axis perpendicular to both the preparation $b$ and
        measurement $a$, so that the integration region is
        defined as a range of allowed polar angles.

        Let $\alpha$ be the angle between $b$ and $a$ and
        choose coordinates such that $b = (0,1,0)$ and $a =
        (-\sin\alpha, \cos\alpha,0)$. If $x =
        (\sin\theta\cos\phi,\sin\theta\sin\phi,\cos\theta)$,
        then $x \in H_b \cap H_b$ if
        \begin{equation}
            b \cdot x
            = \sin\theta\sin\phi > 0,
        \end{equation}
        which implies that $\phi \in (0,\pi)$ (since
        $\sin\theta > 0$ for $\theta \in (0,\pi)$).
        Furthermore, $x$ must satisfy
        \begin{align}
            a \cdot x
            &= -\sin\alpha \sin\theta\cos\phi + \cos\alpha
            \sin\theta\sin\phi \\
            &= \sin\theta \left( 
                -\sin\alpha\cos\phi + \cos\alpha\sin\phi
            \right) \\
            &= \sin\theta \sin(\phi - \alpha) \\
            &> 0,
        \end{align}
        implying that $\phi - \alpha \in (0,\pi)$, i.e.,
        $\phi \in (\alpha,\alpha + \pi)$. Satisfying both
        conditions means that $\phi \in (\alpha, \pi)$.
        Therefore
        \begin{align}
            \int_{H_b \cap H_a} b \cdot x \, d\Omega(x)
            &= \int_{0}^{\pi} \int_{\alpha}^{\pi} \sin\phi
            \sin^2\theta \, d\phi \, d\theta \\
            &= \frac{\pi}{2}
            \int_{\alpha}^{\pi} \sin\phi \, d\phi
            \\
            &= \frac{\pi}{2} \left(
                -\cos(\pi) + \cos(\alpha)
            \right) \\
            &= \frac{\pi}{2} \left( 
                1 + \cos(\alpha)
            \right). 
        \end{align}
        Since $\cos(\alpha) = b \cdot x$ the calculation is
        complete.
        
        This model gives a deterministic explanation for
        spin measurements on a free electron. It does not
        deal with dynamics. Furthermore, a natural question
        arises, is the model measurement non-contextual? At
        least for projection value measures the question
        does not even arise, since the context is fully
        determined once a choice of projector is made.
    }

    So Montina has simply reformulated the KS-model for a
    qubit, which deals with pure states, spin measurements,
    is non?-contextual and does not consider
    transformations. For his communication scenario, the
    model involves the transmission of an infinite amount of
    information, i.e., the unit vector $\vec x_1$ which is
    drawn from a distribution that is \textit{dependent} on
    the preparation $\vec b$. To show how this can be done
    with only two bits of classical information, Montina
    first generalizes the KS-model to include an additional
    hidden variable in the form of another unit vector.

    Define the vectors $\vec y_1 := \vec x_1 + \vec x_2$ and
    $\vec y_2 := \vec x_1 - \vec x_2$.  The probability
    distribution $\rho$ can be obtained from the
    \textit{uniform} distribution
    \begin{equation}
        \rho(\vec x_1, \vec x_2 | \vec b)
        = \frac{1}{4\pi^2} \theta(\vec b \cdot \vec y_1)
        \theta(\vec b \cdot \vec y_2)
    \end{equation}
    by integrating out the unit vector $\vec x_2$. 
    {
        \color{red}
        So we now have a new map from states to probability
        measures over the ontic space $S^2 \times S^2$
        defined by a uniform distribution
        \begin{equation}
            \rho(x_1,x_2 | b)
            = \frac{1}{4\pi^2} \theta(b \cdot y_1) \theta(b
            \cdot y_2),
        \end{equation}
        where $y_1 = x_1 + x_2$ and $y_2 = x_1 - x_2$.
        Marginalization over $x_2$ should recover
        $\rho(\cdot | b)$. First note that the $\rho$ is
        non-zero only when
        \begin{equation}
            b \cdot (x_1 + x_2) > 0
            \quad\text{and}\quad
            b \cdot (x_1 - x_2) > 0,
        \end{equation}
        equivalently
        \begin{equation}
            -b \cdot x_1 < b \cdot x_2 < b \cdot x_1,
        \end{equation}
        for fixed vectors $b$ and $x_1$. Suppose $b =
        (0,0,1)$, and let $\theta_1$ be the angle between
        vectors $b$ and $x_1$, therefore the inequality
        above can be expressed as $-\cos\theta_1 <
        \cos\theta < \cos\theta_1$, which implies that
        $\cos^{-1}(b \cdot x_1) < \theta <
        \cos^{-1}(-b \cdot x_1)$. And so the integral is
        \begin{align}
            \int_{S^2} \rho(x_1,x_2|b) d\Omega(x_2) 
            &= \frac{1}{4\pi^2} \int_{S^2} 
            \theta(b \cdot (x_1 + x_2)) \theta(b \cdot (x_1
            - x_2)) \, d\Omega(x_2) \\
            &= \frac{1}{4\pi^2}
            \int_{\cos^{-1}(b \cdot x_1)}^{\cos^{-1}(-b
            \cdot x_1)}
            2\pi \sin(\theta) \, d\theta \\
            %&= \frac{1}{2\pi} \left[
            %    \cos(\theta)
            %\right]_{-\cos^{-1}(\theta_1)}^{\cos^{-1}(\theta_1)}
            %\\
            &= \frac{1}{2\pi} \left( 
                -(-b \cdot x_1) - -(b \cdot x_1)
            \right) \\
            &= \frac{1}{\pi} \, b \cdot x_1.
        \end{align}
        To define the marginalized function for all of
        $S^2$, we add on the Heaviside function to obtain
        the original distribution
        \begin{equation}
            \rho(x_1 | b)
            = \frac{1}{\pi} \, b \cdot x_1 \, \theta(b \cdot
            x_1)
            \quad
            \text{for all } x_1 \in S^2.
        \end{equation}
        The ontological model now reproduces the Born rule
        by weighting the response function $P(\alpha|x_1)$ 
        by the probability function $\rho(x_1,x_2|b)$ over
        the ontic space $S^2 \times S^2$.
    }

    What has been gained from this is that the ontic space
    is now sampled from a uniform distribution over subsets
    defined by the preparation procedure $b$. We can get
    another exact simulation of a qubit by considering two
    more hidden variables $n_1$ and $n_2$ which are one bit
    each. As we will see, this allows for the sampling of
    the two unit vectors to be \textit{independent} of $b$,
    which for communication purposes permits a cost of just
    the two bits.  Montina defines a probability
    distribution of the two unit vectors conditional on the
    bits, and a response function
    \begin{equation}
        p(\vec x_1, \vec x_2 | n_1, n_2; \vec b)
        = \frac{1}{4\pi^2} \, \theta(n_1 \vec b \cdot \vec
        y_1) \, \theta(n_2 \vec b \cdot \vec y_2),
    \end{equation}
    \begin{equation}
        P(\alpha | \vec x_1, \vec x_2, n_1, n_2; M_A)
        = \theta\left[\vec a_\alpha \cdot (n_1 \vec y_1 +
        n_2 \vec y_2)\right],
    \end{equation}
    for any choice $n_i = \pm 1$. Any particular choice
    defines an ontological model which can be
    derived from the original two vector model above. For
    example with $n_1 = 1$ and $n_2 = -1$, these equations
    can be derived from the distributions above by switching
    $x_1$ and $x_2$. With $n_1 = -1$ and $n_2 = -1$, we have
    to flip the direction of $\vec x_1$ and $\vec x_2$.
    Finally, with $n_1= -1$ and $n_2 = 1$, we have to
    exchange and flip the vectors.

    {
        \color{red}
        Let $n_1 = 1$ and $n_2 = -1$. Then the Montina model
        defines
        \begin{align}
            p(x_1, x_2, | 1, -1, b)
            &= \frac{1}{4\pi^2} \, 
            \theta(b \cdot y_1) \, \theta(-b \cdot y_2) \\
            &= \frac{1}{4\pi^2} \,
            \theta(b \cdot (x_2 + x_1)) \, 
            \theta(b \cdot (x_2 - x_1)),
        \end{align}
        and
        \begin{align}
            P(\alpha | x_1, x_2, 1, -1, M_A)
            &= \theta\left[
                a_{\alpha} \cdot (y_1 - y_2)
            \right] \\
            &= \theta\left[
                a_\alpha \cdot (2 x_2)
            \right] \\
            &= \theta(a_\alpha \cdot x_2).
        \end{align}
        As stated above, this is just the original model
        with the vectors $x_1$ and $x_2$ switched. The other
        options can be similarly derived. The important
        thing is that because of this all of these models
        simulate the qubit exactly.
    }

    Now consider the scenario in which the indices $n_i$ are
    randomly generated with probability $\rho_I(n_1,n_2) = 1
    / 4$. Then the \textit{joint} probability distribution
    of $\vec x_i$ and $n_i$ can be suitably written in the
    form
    \begin{equation}
        \rho(\vec x_1, \vec x_2, n_1, n_2, | \vec b)
        = \rho(n_1, n_2, | \vec x_1, \vec x_2; \vec b)
        \rho_v(\vec x_1, \vec x_2),
    \end{equation}
    where
    \begin{equation}
        \rho(n_1, n_2, | \vec x_1, \vec x_2; \vec b)
        := \theta(n_1 \vec b \cdot \vec y_1) \theta(n_2 \vec
        b \cdot \vec y_2)
    \end{equation}
    and
    \begin{equation}
        \rho_v(\vec x_1, \vec x_2)
        = \frac{1}{(4\pi)^2}.
    \end{equation}
    This model also reproduces the correct statistics but
    also has the property that the marginal probability
    distribution $\rho_v(\vec x_1, \vec x_2)$ of the vectors
    $\vec x_1$ and $\vec x_2$ does not depend on the
    prepared state $\vec b$. The dependence is only in the
    conditional distribution $\rho(n_1,n_2 | \vec x_1, \vec
    x_2; \vec b)$ of the discrete indices $n_i$.

    {
        \color{red}
        The joint probability measure $\rho$ can be
        expressed in two equivalent forms:
        \begin{equation}
            \rho(n_1,n_2,|x_1,x_2,b)\rho_v(x_1,x_2)
            \quad\text{and}\quad
            \rho(x_1,x_2|n_1,n_2,b)\rho_n(n_1,n_2),
        \end{equation}
        for uniform distributions $\rho_v$ and $\rho_n$.
        According to the definition of the distribution of
        the vectors conditioned on the bits, we have
        \begin{align}
            p(x_1,x_2|n_1,n_2,b) p_I(n_1,n_2)
            &= \left(
                \frac{1}{4\pi^2} 
            \theta(n_1 b \cdot y_1)\theta(n_2 b \cdot y_2) 
            \right) \left( \frac{1}{4} \right) \\
            &= \left( \theta(n_1 b \cdot y_1)\theta(n_2 b \cdot
            y_2)\right) \left( 
                \frac{1}{(4\pi)^2}
            \right) \\
            &= p(n_1,n_2|x_1,x_2,b) p_v(x_1,x_2),
        \end{align}
        so
        \begin{equation}
            \rho_v(x_1,x_2)
            = \frac{1}{(4\pi)^2}
            \quad\text{and}\quad
            \rho_n(n_1,n_2)
            = \frac{1}{4}.
        \end{equation}
        Essentially, the randomness has been shifted from
        the unit vectors to the bits, i.e., the joint
        distribution can be expressed as the product of
        distribution of the bits conditional on the unit
        vectors, which are uniformly distributed. The
        important advantage with this expression is the the
        unit vectors do not depend on $b$ anymore. 

        %In the language of ontological models, we have the
        %maps
        %\begin{equation}
        %    \rho : \mathcal{S} \to \mathcal{P}(
        %    S^2 \times S^2 \times \{-1,1\} \times \{-1,1\}
        %    )
        %    \quad\text{and}\quad
        %    P : \mathcal{S} \to \mathcal{F}(\mathcal{M},
        %    \R),
        %\end{equation}
        %where a preparation $b$ defines the probability
        %measure
        %\begin{equation}
        %    \rho(x_1, x_2, n_1, n_2 | b)
        %    = \theta(n_1 b \cdot y_1)\theta(n_2 b \cdot y_2)
        %    \rho_v(x_1, x_2)
        %    = .
        %\end{equation}
    }

    So this procedure is a protocol for simulating the
    communication of a qubit with just two bits of classical
    communication. {\color{blue}
        Bob and Alice share the random vectors $\vec x_1$
        and $\vec x_2$, generated according to the uniform
        probability distribution; given the quantum state
        $\vec b$, Bob generates the discrete indices $n_i$
        according to the rule; he then sends them to Alice,
        whom, given a measurement $M_A$, generates the
        outcome $\vec a_\alpha$ in a deterministic manner.
    } The protocol can be summarized as follows:
    \begin{enumerate}
        \item Alice and Bob hold the same random pair of
            unit vectors $(\vec x_1, \vec x_2) \in S^2
            \times S^2$.
        \item Bob prepares a quantum state according to
            $\vec b$, and deterministically sets the two
            bits according to
            \begin{equation}
                n_1 = \theta(\vec b \cdot \vec y_1)
                \quad\text{and}\quad
                n_2 = \theta(\vec b \cdot \vec y_2),
            \end{equation}
            where $\vec y_1 = \vec x_1 + \vec x_2$ and $\vec
            y_2 = \vec x_1 - \vec x_2$. He then sends the
            two bits to Alice.
        \item Alice outputs the outcome $\alpha$ whose Bloch
            vector maximizes 
            \begin{equation}
                \vec a_\alpha \cdot (n_1 \vec y_1 + n_2 \vec
                y_2).
            \end{equation}
    \end{enumerate}

\end{document}

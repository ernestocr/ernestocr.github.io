\documentclass[a4paper]{article}

\usepackage[margin=1.5in]{geometry}
\usepackage[utf8]{inputenc}
\usepackage[T1]{fontenc}
\usepackage{textcomp}
\usepackage{amsmath}
\usepackage{amssymb}
\usepackage{amsthm}
\usepackage{mathtools}
\usepackage{bm}

%\usepackage{newtxtext,newtxmath}
%\usepackage{tgtermes}
%\usepackage[italic]{mathastext}

%\usepackage{enumitem}
%\usepackage{hyperref}
%\usepackage{graphicx}
%\usepackage{quantikz2}

\DeclareMathOperator{\Tr}{Tr}
\DeclareMathOperator{\N}{\mathbf{N}}
\DeclareMathOperator{\Z}{\mathbf{Z}}
\DeclareMathOperator{\Q}{\mathbf{Q}}
\DeclareMathOperator{\R}{\mathbf{R}}
\DeclareMathOperator{\C}{\mathbf{C}}
\DeclareMathOperator{\F}{\mathbf{F}}

%\DeclarePairedDelimiter\bra{\langle}{\rvert}
%\DeclarePairedDelimiter\ket{\lvert}{\rangle}
%\DeclarePairedDelimiterX\braket[2]{\langle}{
%\rangle}{#1\,\delimsize\vert\,\mathopen{}#2}

%\theoremstyle{definition}
%\newtheorem{defn}{Definition}
%\newtheorem*{sol}{Solution}
%\theoremstyle{plain}
%\newtheorem{thm}{Theorem}
%\newtheorem{lem}{Lemma}
%\newtheorem{cor}{Corollary}
%\newtheorem{exa}{Example}
%\newtheorem{exeinner}{Exercise}
%\newenvironment{exe}[1]{%
%    \IfBlankTF{#1}
%    {\renewcommand{theexeinner}{\unskip}}
%    {\renewcommand\theexeinner{#1}}%
%    \exeinner
%}{\endexeinner}

\title{Process fidelity in the Montina model}
\author{Ernesto Camacho}
\begin{document}
    \maketitle

    \section{What Randomized Benchmarking Actually Measures
    (2017)}

    Randomized benchmarking yields a single error metric
    $r$. For Clifford gates with arbitrary small errors
    described by process matrices, $r$ was believed to
    reliabley correspond to the mean, over all Clifford
    gates, of the average gate infidelity between the
    imperfect gates and ideal gates. The quantity is not a
    well-defined property of a physical gate set.

    RB consists of
    \begin{enumerate}
        \item performing many randomly chosen sequences of
            Clifford gates that ought to return the QIP to
            its initial state,
        \item measuring at the end of each sequence to see
            whether the QIP ``survied'' (i.e., returnd to
            its initial state), and 
        \item plotting the observed survival probabilities
            vs sequence length and fitting this to an
            exponential decay curve.
    \end{enumerate}
    The decay rate of the survival probability is---up to a
    dimensionality constant, and neglecting any
    finite-sampling error---the \textit{RB number} $r$. RB
    experiments estimate $r$, which is used as a metric for
    judging the processor's performance.

    In QIP theory, the ideal target operations and the
    imperfect operations are reprsented by process matrices,
    i.e., CPTP maps. The literature suggests that $r$ is
    approximately equal to the average, over all $n$-qubit
    Clifford gates, of the average gate infidelity (AGI)
    between the imperfect gates and their ideal counterparts
    (AGsI) $\epsilon$. It is believed that $r \approx
    \epsilon$ whenever the errors in the gates are small and
    described by CPTP maps. Proctor et al shows that
    $\epsilon$ is not a well-defined property of a physical
    QIP.

    The standard way to estimate $r$ from RB data is to fit
    the average of the sampled survival probabilities
    $(P_m)$, for many sequence lengths $m$, to the model
    $P_m = A + (B + Cm) p^{m}$, where $A,B,C$ and $p$ are
    fit parameters. The estimate of $p$, denoted $\hat{p}$ 
    gives an estimate of $r$ as 
    \begin{equation}
        \hat{r}
        = \frac{1}{d} (d-1)(1 - \hat{p}),
    \end{equation}
    where $d = 2^{n}$.

    \subsection{The theory of RB}

    The average survival probabilities $P_m$ are real and
    experimentally accessible; $r$ is equally well defined,
    as long as the $P_m$ decay exponentially with $m$. But
    two questions arise naturally. First, under what
    circumstances does $P_m$ decay exponentially? Second,
    when it does, what is $r$? I.e., to what property of the
    imperfect gates does $r$ correspond?

    An as-built processor with an imperfect physical gate
    set can be reprsented by some $\tilde{\mathcal{G}} =
    \{\tilde G_i, \tilde \rho_j, \tilde E_{k,l}\}$ and an
    idealized perfect device by some $\mathcal{G} = \{G_i,
    \rho_j, E_{k,l}\}$. Since $r$ is independent of the
    state preparation and measurement, we usually only need
    representations of the imperfect and ideal Clifford
    gates, $\mathcal{C}$ and $\tilde{\mathcal{C}}$.

    RB theory is clear when the gate set has gate
    independent errors: there is a process matrix $\Lambda$ 
    such that each imperfect CLifford gate can be
    represented as $\tilde C_i = \Lambda C_i$. In this
    situation, $r$ is exactly equal to the AGI between
    $\Lambda$ and the identity process matrix $I$. The AGI
    between process matrices $\tilde G$ and $G$ is simply $1
    - \overline{F}$, where
    \begin{equation}
        \overline{F}\left( \tilde G, G \right) 
        := \int_{\mathcal{S}} \Tr\left( \tilde G(\Pi_\psi)
        G(\Pi_\psi) \right) \, d\psi,
    \end{equation}
    where $\mathcal{S}$ is the set of pure states. But a
    general theory needs to address gate-dependent errors,
    where $\tilde C_i = \Lambda_i C_i$. Observe that for
    gate independent errors, every imperfect Clifford gate
    has teh same AGI with its ideal counterpart,
    $\overline{F}(\tilde C_i, C_i) = \overline{F}(\Lambda,
    I)$. So a plausible generalization of AGI to the
    gate-dependent errors is its average over all Clifford
    gates,
    \begin{equation}
        \epsilon(\tilde{\mathcal{C}}, \mathcal{C})
        := \text{avg}_i [1 - \overline{F}(\tilde C_i, C_i)],
    \end{equation}
    a quantity called AGsI.

    \section{AGI in the KS model}

    The KS model consists of three maps from the operational
    components of (pure state) quantum mechanics and their
    representations in an ontological over some ontic space
    $\Lambda = S^2$. All (one qubit) unitary transformations
    can be seen as rotations in the sphere, so the
    probability distribution corresponding to some state
    preparation $\psi$ simply udpates covariantly, i.e.,
    \begin{equation}
        \mu(x | R_U b)
        = \mu(R_U^{-1} x | b),
    \end{equation}
    where $R_U$ corresponds to the unitary $U$. So it seems
    natural that quantities that measure some sort of
    fidelity between noisy and ideal processes have a
    correspondance in this model. First we need to define
    the notion of \textit{fidelity} in the ontological
    model. We recall that the Born rule is reproduced in the
    KS model as
    \begin{equation}
        \Tr(\Pi_\psi \Pi_\phi)
        = \int_{S^2} \xi(\phi|x) \mu(x|\psi) \, d\Omega(x).
    \end{equation}
    For two pure states $\psi$ and $\phi$, the \textit{state
    fidelity} is simply the transition probability:
    $F(\psi,\phi) = |\langle \psi, \phi \rangle|^2 =
    \Tr(\Pi_\psi \Pi_\phi)$. In the KS model this is
    equivalent to measuring the system prepared by $\psi$ in
    the direction corresponding to $\phi$, i.e.,
    $F(\psi,\phi) = \int_{S^2} \xi(\phi|x) \mu(x|\psi) \,
    d\Omega(x)$. If $U,V \in \text{SU}(2)$, then we can
    calculate the \textit{average gate fidelity}
    \begin{align}
        \overline{F}(U,V)
        &= \Tr\left( U \Pi_\psi U^{\dag} V \Pi_\psi V^{\dag}
        \right) \, d\psi \\
        &= \int_{\mathcal{S}} \int_{S^2} \xi(V\psi | x) \mu(x
        | U\psi) \, d\Omega(x) \, d\psi \\
        &= \int_{\mathcal{S}} F(U\psi,V\psi) \, d\psi.
    \end{align}
    Generalization AGI to gate dependent errors, Proctor et
    al define the \textit{average gate set fidelity}
    \begin{equation}
        \epsilon(\tilde C, C)
        = \text{avg}_i \, \overline{F}(\tilde C_i, C_i),
    \end{equation}
    where $C_i \in \mathcal{C}$ and $\tilde C_i = \Lambda_i
    \circ C_i$ for some error channel, which we will assume
    is unitary.

    %For two unitary channels
    %$E_U(\Pi_\psi) = U \Pi_\psi U^{\dag}$ and $E_V(\Pi_\psi)
    %= V \Pi_\psi V^{\dag}$, we can define the
    %\textit{process fidelity} as
    %\begin{equation}
    %    F(E_U, E_V)
    %    = \frac{1}{d^2} |\Tr(U^{\dag}V)|^2.
    %\end{equation}

    
\end{document}

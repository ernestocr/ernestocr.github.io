\documentclass[12pt,draft]{article}

\usepackage[margin=1.3in]{geometry}
\usepackage[utf8]{inputenc}
\usepackage[T1]{fontenc}
\usepackage{textcomp}
\usepackage{amsmath}
\usepackage{amssymb}
\usepackage{amsthm}
\usepackage{mathtools}
\usepackage{bm}
\usepackage{xcolor}
\usepackage{setspace}

%\usepackage{newtxtext,newtxmath}
%\usepackage{tgtermes}
%\usepackage[italic]{mathastext}

\usepackage{enumitem}
\usepackage{hyperref}
%\usepackage{graphicx}
%\usepackage{quantikz2}

\DeclareMathOperator{\Tr}{Tr}
\DeclareMathOperator{\sgn}{sgn}
\DeclareMathOperator{\N}{\mathbf{N}}
\DeclareMathOperator{\Z}{\mathbf{Z}}
\DeclareMathOperator{\Q}{\mathbf{Q}}
\DeclareMathOperator{\R}{\mathbf{R}}
\DeclareMathOperator{\C}{\mathbf{C}}
\DeclareMathOperator{\F}{\mathbf{F}}

\theoremstyle{definition}
\newtheorem{defn}{Definition}
\newtheorem*{sol}{Solution}
\theoremstyle{plain}
\newtheorem{thm}{Theorem}
\newtheorem{lem}{Lemma}
\newtheorem{cor}{Corollary}
\newtheorem{exa}{Example}

\onehalfspacing

\title{KS-model}
\author{Ernesto Camacho}
\date{28/05/2025}

\begin{document}
    \maketitle

    \section{The Kochen-Specker model}

    In their 1967 paper \textit{The Problem of Hidden
    Variables in Quantum Mechanics}, S. Kochen and E. P.
    Specker prove that no deterministic noncontextual hidden
    variable model can exist that reproduces all of the
    statistics of quantum theory for dimensions three and
    greater. The interesting exception has long been a
    source of disagreement between those that believe that
    the qubit can be viewed as a classical system and those
    that don't. What is true is that deterministic hidden
    variable models for the two-dimensional Hilbert space do
    exist and Kochen and Specker actually give such a model
    in the paper. It is similar in nature to that of John S.
    Bell's qubit model and it has inspired other models like
    Alberto Montina's models for simulating entanglement. It
    is this last reason that we feel the need to go over the
    K-S-model in detail.

    One of the objectives of K-S was to exhibit a classical
    interpretation of a \textit{part} of quantum mechanics
    so that it can be compared with previous (failed)
    attempts of introducing hidden variables for the sake of
    the completion of quantum mechanics. In their
    formulation of the HVM problem, they require the
    specification of a ``phase space'' $\Omega$---which in
    more modern language corresponds to the \textit{ontic
    space} of some ontological model---such that for each
    self-adjoint operator $A$ there is a real-valued
    function $f_A : \Omega \to \R$ and for each state $\psi$
    there exists a probability measure  $\mu_\psi$ on
    $\Omega$ such that
    \begin{enumerate}
        \item $f_{u(A)} = u(f_A)$ for each (Borel) function
            $u$; and
        \item the quantum mechanical average $\langle \psi,
            A\psi\rangle$ is equal to
            \begin{equation}
                \int_\Omega f_A(\omega) \,
                d\mu_\psi(\omega).
            \end{equation}
    \end{enumerate}
    They define the ontic space to be the unit sphere,
    $\Omega = S^2$. They do the analysis for arbitrary
    self-adjoint operator $A$, but it suffices to choose
    spin measurements, and for this, the assignment function
    is defined as
    \begin{equation}
        f_A(p)
        = \begin{cases}
            +1 & \text{for } p \in S^{+}_{P_A} \\
            -1 & \text{otherwise,}
        \end{cases}
    \end{equation}
    where $P_A$ corresponds to the Bloch vector given by
    the spin direction of the measurement and $S^+_{P_A}$
    is the hemisphere defined by the vector $P_A$ pointing
    towards the north pole. K-S state that this definition
    satisfies the functional relations condition (1). 

    Next up is the assignment of a probability measure
    $\mu_\psi$ to each vector $\psi$. Let $P_\psi$ the
    corresponding Bloch vector. Physically, if $\psi$ is the
    state vector of an electron, then the electron is said
    to have ``spin in the direction $P_\psi$''. The authors
    impose some assumptions on the measure to ``simplify''
    the problem. Namely, they assume that $\mu_\psi$ 
    satisfies
    \begin{enumerate}
        \item for each $\psi$, the measure arises from a
            continuous probability density on $u_\psi(p)$ on
            $S^2$, i.e., 
            \begin{equation}
                \mu_\psi(E)
                = \int_{E} u_\psi(p) \, dp,
            \end{equation}
        \item the probability density $u_\psi(p)$ is a
            function only of the angle $\theta$ subtended at
            $0$ by the points $p$ and $P_\psi$ on $S^2$,
        \item if $u(\theta) := u_{\psi_0}(\theta)$ is the
            probability density assigned to $\psi_0 = e_1$,
            then for any $\psi$, if $\alpha$ is the polar
            angle of the point $P_\psi$ on $S^2$, then we
            assume that $u_\psi(\theta) = u(\theta +
            \alpha)$,
        \item assume that $u(\theta) = 0$ for $\theta > \pi
            / 2$.
    \end{enumerate}
    These conditions actually determine the measures
    $\mu_\psi$ uniquely. First of all, for some reason they
    restrict to projections as a consequence of the
    linearity of the value assignment function. Second, they
    take $P_\psi = (0,0,1)$. Furthermore, it is sufficient
    to consider the case where $P_A$ has azimuthal angle
    equal to zero. {\color{blue} Notice that these are all
        the same assumptions made to simplify the integral,
        here K-S justify these assumptions not by some
    geometric invariance but by some model constraints.} 

    Next, they express the expectation value $\langle \psi,
    A\psi\rangle$ as a function of the angle subtended at
    $0$ by the points $P_\psi = (0,0,1)$ and $P_A$, i.e.,
    the polar angle $\rho$ of $P_A$. In spherical
    coordinates we have
    \begin{equation}
        P_A
        = (\sin \rho, 0, \cos \rho).
    \end{equation}
    Since they, for some reason restricted to projection
    operators, the expectation value can be calculated as
    \begin{equation}
        \langle \psi, A\psi \rangle
        = \langle \langle \psi, \eta \rangle\eta, \psi
        \rangle
        = \langle \psi, \eta \rangle \langle \eta, \psi
        \rangle
        = |\langle \psi,\eta \rangle|^2
        = \cos^2(\rho / 2),
    \end{equation}
    where $\eta = (\cos(\rho / 2), \sin(\rho / 2))$ is the
    $+1$ eigenvector of $A$. {\color{blue} Using the double
        angle identity for cosine we see that this
        expectation value is of course equal to what we have
        in Montina's analysis, i.e., 
        \begin{equation}
            \cos^2(\rho / 2)
            = \frac{1}{2} \left( 1 + \cos(\rho) \right)
            = \frac{1}{2} \left( 1 + P_\psi \cdot P_A
            \right). 
        \end{equation}
    }
    They have thus reduced the problem to solving for
    $u(\theta)$ in the integral equation
    \begin{equation}
        \cos^2(\rho / 2)
        = \int_{S^2} f_A(\rho) u(\theta) \, dp.
    \end{equation}
    By definition of the valuation function
    \begin{equation}
        f_A(p)
        = \begin{cases}
            1 & \text{for } p \in S^+_{P_A} \\
            0 & \text{otherwise},
        \end{cases}
    \end{equation}
    the integral equation becomes
    \begin{equation}
        \cos^2(\rho / 2)
        = \int_{T} u(\theta) \, dp,
    \end{equation}
    where
    \begin{equation}
        T
        = S^+_{P_A} \cap S^+_{P_\psi}.
    \end{equation}
    {\color{blue} So once again we have an integral over the
        intersection of two hemispheres, i.e., over a
    spherical lune.} Thus (!)
    \begin{equation}
        \cos^2(\rho / 2)
        = \int_{\rho - \pi / 2}^{\pi / 2}
        \int_{-\phi_\theta}^{\phi_\theta}
        u(\theta) \sin(\theta) \, d\phi \, d\theta
    \end{equation}
    where $\phi_\theta$ is the azimuthal angle of the point
    \begin{equation}
        Q = (\sin\theta \cos\phi_\theta,\sin\theta
        \sin\phi_\theta, \cos\theta)
    \end{equation}
    with polar angle $\theta$ which lies in the great circle
    $C$ perpendicular to the point $P_A = (\sin\rho, 0,
    \cos\rho)$. According to K-S
    \begin{equation}
        \phi_\theta
        = \cos^{-1}(-\cot\rho \cot\theta),
    \end{equation}
    thus
    \begin{equation}
        \cos^2(\rho / 2)
        = 2 \int_{\rho - \pi / 2}^{\pi / 2} 
        u(\theta)\sin(\theta) \cos^{-1}(-\cot\rho
        \cot\theta) \, d\theta.
    \end{equation}
    By some complicated calculations that involve Abel's
    integral equation which requires Laplace transforms to
    solve, they arrive at the conclusion that
    \begin{equation}
        u(\theta)
        = \begin{cases}
            \frac{1}{\pi} \cos(\theta) & \text{if } 0 \leq
            \theta \leq \pi / 2 \\
            0 & \text{otherwise.}
        \end{cases}
    \end{equation}
    {\color{blue} This is the same as Montina's model! I
    would still like to solve the integral over the
    spherical lune directly.} Ignoring the non-triviality of
    the integral, we see that by construction integrating
    $u(\theta)$ over the lune reproduces the Born rule.
     
    Therefore K-S are now in a position to construct a
    simple classical model of electron spin, which by
    linearity extends to all observables. 

    Start with a unit sphere centered at the origin $O$. A
    point $P$ on the sphere represents the quantum state
    ``spin in the direction $P$'' (i.e., the Bloch vector).
    If the sphere is in such a quantum state, it is at the
    same time in a hidden state which is represented by
    another point $T \in S_P^+$. The point $T$ is determined
    in the following manner. A disk $D$ of the same radius
    as the sphere is place perpendicular to the $P$ axis
    with center directly above $P$. A particle is placed on
    the disk and the disk is shaken randomly, i.e., the disk
    is shaken so that the probability of the particle being
    in a region $U$ in $D$ is proportional to the area of
    $U$ (i.e., the probability is uniformly distributed).
    The point $T$ is the orthogonal projection of the
    particle (after shaking) onto the sphere. It is easily
    seen that the probability density function for the
    projection is given by 
    \begin{equation}
        u(T)
        = \begin{cases}
            \frac{1}{\pi} \cos\theta & \text{if } 0 \leq
            \theta \leq \pi / 2 \\
            0 & \text{otherwise,}
        \end{cases}
    \end{equation}
    where $\theta$ is the angle subtended by $T$ and $P$ at
    $O$.

    If we now wish to measure the spin angular momentum in
    direction $Q$, it can be determined as follows. If $T
    \in S^+_Q$, then the spin angular momentum is $+\hbar /
    2$, if  $T \not\in S^+_Q$ then the spin is $-\hbar / 2$.
    The sphere is now in the new quantum state of spin in
    the direction $Q$ if $T \in S^+_Q$ or spin in the
    direction $Q^{*}$ (the antipodal point of $Q$) if $T
    \not \in S^+_Q$. The new hidden state of the sphere is
    now determined as before, by shaking the particle on the
    disk $D$, the disk being place with center above $Q$ if
    $T \in S^+_Q$ or with center above $Q^{*}$ otherwise.
    {\color{blue} The important thing to note here is the
        dependence of the hidden state $T$ on the
        preparation procedure $P$. It seems that Molina's
        innovation is to move the dependence on the
        preparation from the continuous unit vector to the
        discrete and finite bits. But he does this for
        communication reasons in his simulation
        investigation, we are more interested in the
        relationship with the DWF.
    } The probabilities and expectation that arise in this
    model match those of the quantum mechanics for the free
    electron spin. The probabilities arise throught
    ignorance of the observer of the sphere of the actual
    location of the particle on the disk. To an observer of
    the complete system of sphere and disk the model is a
    deterministic classical system. K-S then go on to
    analyze von Neumann's proof of the impossiblity of
    hidden variables and show, of course, how their model
    evades the assumptions made by von Neumann.

    \subsection{Ryan Morris}
    {
        \color{blue}
        This is the proof Ryan Morris gives in his thesis,
        which is basically a reproduction of the original KS
        proof. In the language of ontological models used in
        Morris' work, the KS-model is a $\psi$-epistemic
        model for the pure qubit states. The ontic space
        $\Lambda$ is $S^2$, which is isomorphic to to
        $\bm{CP}^{1}$ the space of pure states. 

        Denote by $\lambda(\psi)$ the ontic state
        corresponding to the pure state $\psi$, which is
        given by its Bloch vector on $S^2$. For a given
        preparation $\psi$, the map $\mu$ is defined as
        \begin{equation}
            \mu_{\psi}(\lambda)
            = 
            \begin{cases}
                \frac{1}{\pi} \cos(\theta) & 0 \leq \theta
                \leq \frac{\pi}{2}, \\
                0 & \text{otherwise},
            \end{cases}
        \end{equation}
        where $\theta$ is the angle between $\lambda$ and
        $\lambda(\psi)$. The response function representing
        a PVM projector onto $\phi$ is given by
        \begin{equation}
            \xi_{\phi}(\lambda)
            = 
            \begin{cases}
                1 & 0 \leq \theta \leq \frac{\pi}{2}, \\
                0 & \text{otherwise}.
            \end{cases}
        \end{equation}
        To reproduce the statistics given by the Born rule
        for state $\psi$ and measurment given by $\phi$,
        whose angle separation is $\theta$, we need to show
        that
        \begin{equation}
            \cos^2\left( \frac{\theta}{2} \right) 
            = \int_{\Lambda} \xi_\phi(\lambda)
            \mu_\psi(\lambda)  \, d\lambda.
        \end{equation}
        By rotational invariance, it sufficies to show this
        for the particular state $\psi$ with Bloch vector
        $(0,0,1)$, that is, the $+1$-eigenstate of the $Z$ 
        operator, and the projector corresponding to the
        state with Bloch vector $(0, \sin(\theta),
        \cos(\theta)$. Morris denotes this vector by its
        polar and azimuthal angles $\psi_{\theta,\pi / 2}$,
        and note that it lies in the $zy$-plane. 

        The integral is actually non trivial unless you
        choose very convenient parametrizations of the
        sphere. The key is actually to use two distinct
        parametrizations. One is the usual $(\theta,\phi)$ 
        used above where $\theta$ is the polar angle
        descending from the $z$-axis and $\phi$ is the
        azimuthal angle around the $z$-axis starting with
        at $x$-axis. The second parametrization that Ryan
        uses is denoted by $(\beta,\alpha)$ where $\beta$ is
        the polar angle descending from $x$-axis and
        $\alpha$ is the azimuthal angle around the $x$-axis
        starting at the $y$-axis.

        ...
    }

    Ryan states that the KS-model can be extended to a
    convex ontological model for all two-dimensional quantum
    theory. Mixed preparations are accounted for by
    extending the map $\mu$ linearly, and for mixed
    measurements (proper $2$-level POVMs) we extend $\xi$.
    But, the KS-model cannot reproduce all of the statistics
    for all improper $2$-level POVMs, in particular Morris
    shows this to be the case for the \textit{trine POVM}.
\end{document}

\documentclass[a4paper]{article}

\usepackage[margin=1.5in]{geometry}
\usepackage[utf8]{inputenc}
\usepackage[T1]{fontenc}
\usepackage{textcomp}
\usepackage{amsmath}
\usepackage{amssymb}
\usepackage{amsthm}
\usepackage{mathtools}
\usepackage{bm}

%\usepackage{newtxtext,newtxmath}
%\usepackage{tgtermes}
%\usepackage[italic]{mathastext}

%\usepackage{enumitem}
%\usepackage{hyperref}
%\usepackage{graphicx}
%\usepackage{quantikz2}

\DeclareMathOperator{\N}{\mathbf{N}}
\DeclareMathOperator{\Z}{\mathbf{Z}}
\DeclareMathOperator{\Q}{\mathbf{Q}}
\DeclareMathOperator{\R}{\mathbf{R}}
\DeclareMathOperator{\C}{\mathbf{C}}
\DeclareMathOperator{\F}{\mathbf{F}}

%\DeclarePairedDelimiter\bra{\langle}{\rvert}
%\DeclarePairedDelimiter\ket{\lvert}{\rangle}
%\DeclarePairedDelimiterX\braket[2]{\langle}{
%\rangle}{#1\,\delimsize\vert\,\mathopen{}#2}

\theoremstyle{definition}
\newtheorem{defn}{Definition}
\newtheorem*{sol}{Solution}
\theoremstyle{plain}
\newtheorem{thm}{Theorem}
\newtheorem{lem}{Lemma}
\newtheorem{cor}{Corollary}
\newtheorem{exa}{Example}
\newtheorem{prop}{prop}
%\newtheorem{exeinner}{Exercise}
%\newenvironment{exe}[1]{%
%    \IfBlankTF{#1}
%    {\renewcommand{theexeinner}{\unskip}}
%    {\renewcommand\theexeinner{#1}}%
%    \exeinner
%}{\endexeinner}

\title{The Berry Phase}
\author{Ernesto Camacho}
\begin{document}
    \maketitle

    Applications/manifestations:
    \begin{itemize}
        \item The passage of photons in optical fibres
        \item The spectra of molecules
        \item The quantum Hall effect
        \item Anomalies in quantum field theory
    \end{itemize}

    In 1983 Barry Simon observed that Berry's phase can be
    seen as a geometric object, namely, as the holonomy in a
    certain fibre bundle, known as the \textit{spectral
    bundle}. Of course, the theory of fibre bundles was
    successfully applied in the mathematical formulation of
    guage theories.

    Quantum phsics also shows its intricate beauty when one
    applies an appropriate geometric framework.

    \begin{defn}
        A topological space $M$ is called an $n$-dimensional
        \textit{topological manifold} if it looks locally
        like a Euclidean space $\R^{n}$, or, more precisely
        if there exists a family of open subsets $\{U_i\}_{i
        \in I}$ of $M$ such that
        \begin{itemize}
            \item It covers $M$, i.e., $\cup_{i \in I} U_i =
                M$. 
            \item For each $i \in I$ there is a
                homeomorphism (continuous, invertible map)
                \begin{equation}
                    \phi_i
                    : U_i \to \phi(U_i) \subset \R^{n},
                \end{equation}
                where the pair $(U_i,\phi_i)$ is called a
                chart. Moreover, any such two charts have to
                be compatible, i.e., the map
                \begin{equation}
                    \phi_{ji}
                    := \phi_j \circ \phi_i^{-1}
                    : \phi_i(U_i \cap U_j)
                    \to \phi_j(U_i \cap U_j),
                \end{equation}
                must be a homeomorphism.
        \end{itemize}
        A collection of compatible charts covering $M$ is
        called an atlas. If the maps $\phi_{ji}$ are $C^{N}$ 
        diffeomorphisms, then $M$ is called a
        $C^{N}$-manifold.
    \end{defn}

    An atlas on a manifold $M$ gives us differential
    calculus. A function $f : M \to \R^{n}$ is
    differentiable if
    \begin{equation}
        f \circ \phi_i^{-1} : \phi_i(U_i) \to \R^{n}
    \end{equation}
    is differentiable for each chart $(U_i,\phi_i)$. The
    space of smooth functions on $M$ is denoted by
    $C^{\infty}(M)$. 

    \begin{defn}
        At each point $x \in U \subset M$ we can define the
        differential operator $V_x$ by
        \begin{equation}
            V_x(f)
            := \sum_{i=1}^{n} V_x^{i} \frac{\partial
            f}{\partial x^{i}}(x),
        \end{equation}
        for any $f \in C^{\infty}(M)$. These differential
        operators span an $n$-dimensional linear space
        called the tangent space $T_xM$ at the point $x$.
        The set
        \begin{equation}
            \left\{
                \frac{\partial}{\partial x^{1}}\big|_x,
                \ldots,
                \frac{\partial}{\partial x^{n}}\big|_x
            \right\}
        \end{equation}
        form a basis for $T_xM$, called the coordinate
        basis.
    \end{defn}

    \begin{defn}
        This will be better formalized later on, but for
        now, an assignment
        \begin{equation}
            M \ni x \to V_x \in T_xM,
        \end{equation}
        is called a vector field. The space of vector fields
        on $M$ is a $C^{\infty}(M)$-module denoted by
        $\mathfrak{X}(M)$.
    \end{defn}

    \begin{defn}
        Denote by $T_x^{*}M$ the algebraic dual of $T_xM$,
        called a \textit{cotangent space} at $x$. The
        coordinate basis in $T_xM$ induces the dual basis in
        $T_x^{*}M$:
        \begin{equation}
            \left\{
                dx^{1},\ldots,dx^{n}
            \right\},
        \end{equation}
        such that
        \begin{equation}
            dx^{i} \left( \frac{\partial}{\partial x^{j}}
            \right) = \delta^{i}{}_{j}.
        \end{equation}
    \end{defn}
    
    We can generalize the concept of a vector field to that
    of covector fields, and even more generally, to the
    concept of a \textit{tensor field} on $M$.

    \begin{defn}
        The smooth map
        \begin{equation}
            M \ni x \to T(x) \in T_x^{(k,l)}M
            =: \underbrace{T_xM \otimes \cdots \otimes
            T_xM}_{\text{$k$ times}} \otimes 
            \underbrace{T_x^{*}M \otimes \cdots \otimes
            T_x^{*}M}_{\text{$l$ times}},
        \end{equation}
        is called a tensor field of type $(k,l)$. Using a
        coordinate basis we have
        \begin{equation}
            T
            = T^{i_1\ldots i_k}{}_{j_1\ldots j_k} 
            \frac{\partial}{\partial x^{i_1}} \otimes \cdots
            \otimes \frac{\partial}{\partial x^{i_k}}
            \otimes dx^{j_1} \otimes \cdots \otimes
            dx^{j_l}.
        \end{equation}
    \end{defn}

    \begin{defn}
        Let $\phi : M \to N$ be a smooth map between to two
        manifolds $M$ and $N$. The derivative $T_x\phi$ of
        $\phi$ at a point $x \in M$ is a linear map
        \begin{equation}
            T_x \phi
            : T_xM \to T_{\phi(x)}N,
        \end{equation}
        such that
        \begin{equation}
            [T_x\phi(v_x)](f)
            := v_x(\phi \circ f),
        \end{equation}
        for any $f \in C^{\infty}(N)$ and $v_x \in T_xM$.
    \end{defn}

    Using the map $\phi : M \to N$ one may transport tensor
    fields between $M$ and $N$. 

    \begin{defn}
        Let $\omega$ be a tensor field of type $(0,k)$ on
        $N$. A \textit{pull-back} $\phi^{*}\omega$ of
        $\omega$ is a $(0,k)$-tensor field on $M$ defined by
        \begin{equation}
            (\phi^{*}\omega)_x(v_1,\ldots,v_k)
            :=
            \omega_{\phi(x)}(T_x\phi(v_1),\ldots,T_x\phi(v_k))
        \end{equation}
        for any $x \in M$ and $v_1,\ldots,v_k \in T_xM$.
    \end{defn}

    \begin{defn}
        A tensor field $U$ of type $(l,0)$ may be
        \textit{pushed forward} from $M$ and $N$, giving
        rise to an $(l,0)$-tensorfield $\phi_*U:$
         \begin{equation}
             (\phi_*U)_{\phi(x)}(\alpha_1,\ldots,\alpha_l)
             :=
             U_x(\phi^{*}_x\alpha_1,\ldots,\phi^{*}_x\alpha_l),
        \end{equation}
        where $\alpha_1,\ldots,\alpha_l \in
        T_{\phi(x)}^{*}N$.
    \end{defn}

    Of course, if $\phi$ is a diffeomorphism, then $\phi_* =
    (\phi^{-1})^{*}$.

    An important class of tensor fields on a differentiable
    manifold that play an important role in physical
    applications are the differential forms.

    \begin{defn}
        A skew-symmetric tensor of type $(0,k)$ is called a
        differential form of order $k$ (or simply a
        $k$-form).
    \end{defn}

    Denote by $\Lambda^{k}(M)$ the space of $k$-forms on
    $M$. Evidently, $\Lambda^{k}(M) = \{0\}$ for $k > n$.
    The space of differential forms on $M$, denoted by
    $\Lambda(M)$, splits into the following direct sum:
    \begin{equation}
        \Lambda(M)
        = \bigoplus_{k=0}^{n} \Lambda^{k}(M),
    \end{equation}
    with $\Lambda^{0}(M) = C^{\infty}(M)$. The space
    $\Lambda(M)$ is equipped with two basic operations, the
    \textit{wedge product} and the \textit{exterior
    derivative}.

    \begin{defn}
        A wedge product $\wedge$ (also called the exterior
        or Grassmann product):
         \begin{equation}
            \wedge
            : \Lambda^{k}(M) \times \Lambda^{l}(M)
            \to \Lambda^{k+l}(M),
        \end{equation}
        is defined by
        \begin{equation}
            \alpha \wedge \beta
            := \frac{(k+l)!}{k!l!} \bf A (\alpha \otimes
            \beta),
        \end{equation}
        where $\bf A$ is the alternation operator which
        selects the skew-symmetric part of the
        $(0,k+l)$-tensor $\alpha \otimes \beta$.
    \end{defn}
    In terms of local coordinates $(x^{1},\ldots,x^{n})$ any
    $k$-form $\alpha$ has the following component
    representation:
    \begin{equation}
        \alpha
        = \frac{1}{k!} \alpha_{i_1\ldots i_k}
        dx^{i_1} \wedge \cdots \wedge dx^{i_k}.
    \end{equation}
    We can show that
    \begin{equation}
        (\alpha \wedge \beta)_{i_1\ldots i_k}
        = \alpha_{[i_1\ldots i_k} \beta_{i_{k+1}\ldots
        i_{k+l}]},
    \end{equation}
    where the square bracket stands for anti-symmetrization.

    \begin{prop}
        The wedge product satisfies
        \begin{itemize}
            \item $(\alpha \wedge \beta) \wedge \gamma =
                \alpha \wedge (\beta \wedge \gamma)$,
            \item $\alpha \wedge \beta = (-1)^{kl} \beta
                \wedge \alpha$, where $\alpha \in
                \Lambda^{k}(M)$ and $\beta \in
                \Lambda^{l}(M)$.
        \end{itemize}
    \end{prop}

    \begin{defn}
        The \textit{interior product} is a contraction of a
        vector field $v \in \mathfrak{X}(M)$ and a $k$-form
        $\alpha$:
        \begin{equation}
            i_v
            : \Lambda^{k}(M)
            \to \Lambda^{k-1}(M),
        \end{equation}
        that is,
        \begin{equation}
            (i_v \alpha)_{i_1\ldots i_{k-1}}
            := v^{j} \alpha_{j_1\ldots i_{k-1}}.
        \end{equation}
    \end{defn}

    \begin{defn}
        The exterior derivative
        \begin{equation}
            d
            : \Lambda^{k}(M)
            \to \Lambda^{k+1}(M)    
        \end{equation}
        is defined as follows:
        \begin{equation}
            d\alpha
            = \frac{1}{k!} \frac{\partial \alpha_{i_1\ldots
            i_k}}{\partial x^{j}} dx^{j} \wedge dx^{i_1}
            \wedge \cdots \wedge dx^{i_k},
        \end{equation}
        for any $k$-form $\alpha$ represented as such in
        local coordinates.
    \end{defn}
    Since a function $f \in \Lambda^{0}(M)$ is a $0$-form,
    one defines its exterior derivative to be the $1$-form:
    \begin{equation}
        df
        = \frac{\partial f}{\partial x^{i}} dx^{i}.
    \end{equation}
    If $M = \R^{n}$ and $(x^{1},\ldots,x^{n})$ are cartesian
    coordinates, then $df$ reproduces the components of
    $\text{grad }f$. Similarly, the components of the
    exterior derivative of a $1$-form in cartesian
    coordinates reproduces the components of the curl of a
    vector field. And the exterior derivative of a $2$-form
    represents the divergence of a vector field.

    \begin{prop}
        The exterior derivative satisfies
        \begin{itemize}
            \item $d^2\alpha = d(d\alpha) = 0$ for any
                $\alpha \in \Lambda(M)$.
            \item $d(\alpha \wedge \beta) = d\alpha \wedge
                \beta + (-1)^{k} \alpha \wedge d\beta$ for
                $\alpha \in \Lambda^{k}(M)$.
        \end{itemize}
    \end{prop}

    For any smooth map $\phi : M \to N$, the pull-back
    induces a map
    \begin{equation}
        \phi^{*} : \Lambda(N) \to \Lambda(M)
    \end{equation}
    between differential forms.

    \begin{prop}
        The pull-back operation commutes with wedge product
        and exterior derivative, i.e.,
        \begin{equation}
            \phi^{*}(\alpha \wedge \beta)
            = \phi^{*}\alpha \wedge \phi^{*}\beta,
        \end{equation}
        and
        \begin{equation}
            \phi^{*}(d\alpha)
            = d(\phi^{*}\alpha),
        \end{equation}
        for any differential forms $\alpha$ and $\beta$ on
        $N$. 
    \end{prop}

    To perform $n$-dimensional integration one needs
    $n$-forms.

    \begin{defn}
        An $n$-dimensional manifold $M$ is orientable if and
        only if there exists a nowhere vanishing $n$-form
        $\tau$ on it.
    \end{defn}

    \begin{defn}
        A Riemannian manifold $(M,g)$ is a smooth manifold
        $M$ together with a smotth tensor $g$ of type
        $(0,2)$, called a metric tensor, such that
        \begin{itemize}
            \item $g$ is symmetric,
            \item for each $x \in M$, the bilinear form $g_x
                : T_xM \times T_xM \to \R$ is nondegenerate.
        \end{itemize}
    \end{defn}


\end{document}

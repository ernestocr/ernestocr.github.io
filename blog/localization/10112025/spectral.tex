\documentclass[a4paper]{article}

\usepackage[margin=1.5in]{geometry}
\usepackage[utf8]{inputenc}
\usepackage[T1]{fontenc}
\usepackage{textcomp}
\usepackage{amsmath}
\usepackage{amssymb}
\usepackage{amsthm}
\usepackage{mathtools}
\usepackage{bm}

%\usepackage{newtxtext,newtxmath}
%\usepackage{tgtermes}
%\usepackage[italic]{mathastext}

%\usepackage{enumitem}
\usepackage{hyperref}
%\usepackage{graphicx}
%\usepackage{quantikz2}

\DeclareMathOperator{\Tr}{Tr}
\DeclareMathOperator{\spn}{span}
\DeclareMathOperator{\ran}{ran}
\DeclareMathOperator{\dom}{dom}

\DeclareMathOperator{\N}{\mathbf{N}}
\DeclareMathOperator{\Z}{\mathbf{Z}}
\DeclareMathOperator{\Q}{\mathbf{Q}}
\DeclareMathOperator{\R}{\mathbf{R}}
\DeclareMathOperator{\C}{\mathbf{C}}
\DeclareMathOperator{\F}{\mathbf{F}}

\DeclareMathOperator{\pp}{pp}
\DeclareMathOperator{\ac}{ac}
\DeclareMathOperator{\sgc}{sc}

%\DeclarePairedDelimiter\bra{\langle}{\rvert}
%\DeclarePairedDelimiter\ket{\lvert}{\rangle}
%\DeclarePairedDelimiterX\braket[2]{\langle}{
%\rangle}{#1\,\delimsize\vert\,\mathopen{}#2}

%\theoremstyle{definition}
%\newtheorem{defn}{Definition}
%\newtheorem*{sol}{Solution}
%\theoremstyle{plain}
%\newtheorem{thm}{Theorem}
%\newtheorem{lem}{Lemma}
%\newtheorem{cor}{Corollary}
%\newtheorem{exa}{Example}
%\newtheorem{exeinner}{Exercise}
%\newenvironment{exe}[1]{%
%    \IfBlankTF{#1}
%    {\renewcommand{theexeinner}{\unskip}}
%    {\renewcommand\theexeinner{#1}}%
%    \exeinner
%}{\endexeinner}

\title{A brief note on operator spectra}
\author{Ernesto}

\begin{document}
    \maketitle

    I keep forgetting basic definitions pertaining to the
    spectral theory of self-adjoint operators so I want to
    keep a few of them here as a reminder. I'll illustrate
    the concepts with examples arising in the theory of
    Anderson localization.

    A confusion that can occur especially when studying
    the physics literature, is that of mixing up the
    various definitions and nomenclature related to the
    spectrum of an operator.  For example, the language
    usually used could leave one confused when meeting an
    operator $A$ with its spectrum equal to some continuous
    subset of the real line, yet having only a ``pure point
    spectrum''.  Additionally, it is common to say an
    operator has pure point spectrum for both the case in
    which it is entirely pure point or when it only has
    non-trivial pure point part (like the Hamiltonian of
    the hydrogen atom), but not much can be done about this
    except to pay attention to the context we're in. This
    post is just a brief recollection of these concepts.

    \section{Bare minimum spectral theory}

    What follows can be found in detail in M. Aizenman and
    S.  Warzel's \textit{Random Operators}. A lot of the
    following definitions apply to some broader class of
    linear operators, but we will implicitly assume that we
    deal with self-adjoint operators. Let $A :
    \mathcal{D}(A) \to \mathcal{H}$ be a densely defined
    self-adjoint operator on some Hilbert space
    $\mathcal{H}$. The \textit{resolvent set} is defined as
    \begin{equation}
        \rho(A)
        := \{z \in \C : (A - z I) : \mathcal{D}(A) \to
        \mathcal{H} \text{ is bijective}\},
    \end{equation}
    and the inverse $(A - z I)^{-1}$ is called \textit{the
    resolvent}. The \textit{spectrum} of $A$ is $\sigma(A)
    := \C \setminus \rho(A)$, which is a closed subset of
    the real line. It can be shown that for any $\psi \in
    \mathcal{H}$ the function defined by
    \begin{equation}
        F_\psi(z)
        := \langle \psi, (A-zI)^{-1} \psi \rangle
    \end{equation}
    is a \textit{Herglotz-Pick} function, which means that
    it is a holomorphic function $F_\psi : \C^{+} \to
    \C^{+}$ from the upper-half plane into itself. There is
    a representation theorem for these types of functions
    which states that there exists a unique finite Borel
    measure $\mu_\psi$ on $\R$ called the \textit{spectral
    measure of $A$} associated to $\psi$, which satisfies
    \begin{equation}
        F_\psi(z)
        = \int_{\R} \frac{1}{\lambda - z} \,
        \mu_\psi(d\lambda).
    \end{equation} 
    By polarization we obtain a complex finite Borel measure
    $\mu_{\phi,\psi}$, which can be used to define a
    functional calculus for bounded functions $f \in
    L^{\infty}(\R)$ satisfying
    \begin{equation}
        \langle \phi, f(A) \psi \rangle
        = \int_{\R} f(\lambda) \, \mu_{\phi,\psi}(d\lambda),
        \quad
        \forall \phi,\psi \in \mathcal{H}.
    \end{equation}
    Of course the characteristic functions $\chi_{B}$ are
    bounded for any Borel set $B \in \mathcal{B}$ and can be
    used to define orthogonal projections $P_A(B)$. The map
    $P_A : \mathcal{B} \to \mathcal{L}(\mathcal{H})$ defined
    in this manner is a \textit{projection-valued measure}.
    The spectral theorem guarantees a correspondence
    between these projection-valued measures and
    self-adjoint operators on $\mathcal{H}$. In particular
    we have $\mu_{\phi,\psi}(B) = \langle \phi, P_A(B) \psi
    \rangle$.

    Having such a Borel measure $\mu$, \href{
        https://en.wikipedia.org/wiki/Lebesgue%27s_decomposition_theorem
    }{Lebesgue's decomposition theorem} states that $\mu$
    can be decomposed with respect to the Lebesgue measure
    into three mutually singular parts:
    \begin{equation}
        \mu
        = \mu^{\pp} + \mu^{\ac} + \mu^{\sgc},
    \end{equation}
    known as the \textit{pure point} part, the
    \textit{absolutely continuous} part and the
    \textit{singular continuous} part, respectively.
    Occasionally one calls these parts, spectral components.
    Using these components, we can define the closed
    subspaces
    \begin{equation}
        \mathcal{H}^{\#}
        = \left\{
            \psi \in \mathcal{H} : \mu_\psi =
            \mu_\psi^{\#}
        \right\},
    \end{equation}
    where $\#$ is $\pp, \ac$ or $\sgc$. It turns out that
    the Hilbert space can be decomposed as a direct sum of
    these subspaces:
    \begin{equation}
        \mathcal{H}
        = \mathcal{H}^{\pp} \oplus
        \mathcal{H}^{\ac} \oplus
        \mathcal{H}^{\sgc},
    \end{equation}
    with each summand being invariant under $A$. If $A$ is
    bounded, then restricting to these subspaces defines the
    \textit{components} of the spectrum
    \begin{equation}
        \sigma^{\#}(A)
        = \sigma\left( A\big|_{\mathcal{H}^{\#}} \right),
    \end{equation}
    where $\sigma^{\#}(A)$ is the \textit{pure point},
    \textit{absolutely continuous} or \textit{singular
    continuous} spectrum of $A$. 

    From a slightly different perspective, we can define
    the \textit{discrete spectrum} $\sigma_{\text{dis}}(A)$
    as the set of isolated eigenvalues of finite
    multiplicity of $A$. An isolated eigenvalue $\lambda$ of
    $A$ is one which there exists $\epsilon > 0$ such that
    $\sigma(A) \cap (\lambda - \epsilon, \lambda + \epsilon)
    = \{\lambda\}$ and any isolated point in the spectrum is
    always an eigenvalue. The complement of
    $\sigma_{\text{dis}}(A)$ in $\sigma(A)$, denoted by
    $\sigma_{\text{ess}}(A)$, is called the
    \textit{essential spectrum}. It can easily be shown that
    the pure point spectrum is equal to the closure of the
    set of eigenvalues of $A$ and so
    $\sigma_{\text{dis}}(A) \subset \sigma^{\pp}(A)$.

    We can produce a finer
    \href{https://en.wikipedia.org/wiki/Spectrum_(functional_analysis)#Classification_of_points_in_the_spectrum}{classification
    of spectra}, and
    even more so for an unbounded operator, but we stop here
    since this is enough to already confuse oneself and is
    sufficient for the few examples that follow.

    \section{Illustration}

    Consider the graph (also discrete, finite-difference)
    Laplacian $\Delta : \mathcal{D}(\Delta) \to
    \ell^2(G)$ for $G = \Z^{d}$. For a function $\psi \in
    \mathcal{D}(\Delta)$ it is defined as
    \begin{equation}
        (\Delta \psi)(x)
        = \sum_{y : |x-y|=1}^{} 
        (\psi(y) - \psi(x)).
    \end{equation}
    This is a bounded operator with $\|\Delta\| = 4d$. We
    can compute its spectrum easily by using the Fourier
    transform $F : \ell^2(\Z^{d}) \to L^2([0,2\pi]^{d})$ 
    which is defined as
    \begin{equation}
        (F\psi)(k)
        = (2\pi)^{-d / 2} 
        \sum_{x \in \Z^{d}}^{} e^{-i k \cdot x} \psi(x).
    \end{equation}
    Then by direct computation we can show for any $\phi \in
    L^2([0,2\pi]^{d})$ that
    \begin{align}
        (F \Delta & F^{-1} \phi)(k) \\
        %&= (2\pi)^{-d / 2} \sum_{x \in \Z^{d}}^{} 
        %e^{-i k \cdot x} (\Delta F^{-1} \phi)(x) \\
        &= (2\pi)^{-d / 2} \sum_{x \in \Z^{d}}^{} 
        e^{-i k \cdot x} \left\{ 
            \sum_{y : |x-y| = 1}^{} \left[
                (F^{-1}\phi)(y) - (F^{-1}\phi)(x)
            \right]
        \right\} \\
        %&= (2\pi)^{-d / 2} \sum_{x \in \Z^{d}}^{} 
        %e^{-i k \cdot x} \left\{
        %    \sum_{j=1}^{d} \left[
        %        (F^{-1}\phi)(x + e_j)
        %        + (F^{-1}\phi)(x - e_j)
        %    \right] - 2d (F^{-1}\phi)(x)
        %\right\} \\
        %&= (2\pi)^{-d} \sum_{x \in \Z^{d}}^{} 
        %e^{-i k \cdot x}
        %\left\{
        %    \sum_{j=1}^{d} \left[
        %        \int_{[0,2\pi]^{d}} e^{i w \cdot (x+e_j)}
        %        \phi(w) \, dw
        %        +
        %        \int_{[0,2\pi]^{d}} e^{i w \cdot (x-e_j)}
        %        \phi(w) \, dw
        %    \right]
        %    - 2d \int_{[0,2\pi]^{d}} e^{iw \cdot x} \phi(w)
        %    \, dw
        %\right\} \\
        %&= (2\pi)^{-d} \int_{[0,2\pi]^{d}} \sum_{x \in
        %\Z^{d}}^{} \left\{ e^{-i k \cdot x} 
        %\sum_{j=1}^{d}
        %\left[e^{iw \cdot (x + e_j)} + e^{iw
        %\cdot (x - e_j)}\right] \phi(w)
        %- 2d e^{i w \cdot x} \phi(w) 
        %\right\} \, dw \\
        %&= (2\pi)^{-d} \int_{[0,2\pi]^{d}} \sum_{x \in
        %\Z^{d}}^{} \left\{ e^{-i k \cdot x} 
        %\sum_{j=1}^{d}
        %\left[
        %    e^{iw \cdot x} (e^{iw \cdot e_j} + e^{-iw \cdot
        %    e_j})
        %\right] \phi(w)
        %- 2d e^{i w \cdot x} \phi(w) 
        %\right\} \, dw \\
        &= (2\pi)^{-d} \int_{[0,2\pi]^{d}} \sum_{x \in
        \Z^{d}}^{} \left\{ e^{-i k \cdot x} e^{i w \cdot x}
        \left(
            \sum_{j=1}^{d}
            \left[
                2 \cos(w \cdot e_j)
            \right]
            - 2d 
        \right) \phi(w)
        \right\} \, dw \\
        %&= (2\pi)^{-d} \int_{[0,2\pi]^{d}}
        %2 \left( 
        %    \sum_{j=1}^{d}
        %    \left[
        %        \cos(w \cdot e_j)
        %    \right]
        %    - d
        %\right) \left(
        %    \sum_{x \in \Z^{d}}^{} e^{i(w - k) \cdot x} 
        %\right) \phi(w) \, dw \\
        %&= \int_{[0,2\pi]^2} 2 \sum_{j=1}^{d} [\cos(w \cdot
        %e_j) - 1] \delta(w-k) \phi(w) \, dw \\
        &= \left(
            2 \sum_{j=1}^{d} [\cos(k \cdot e_j) - 1]
        \right) \phi(k).
    \end{align}
    Thus $\Delta$ is unitarily equivalent to a
    multiplication operator on $L^2([0,2\pi]^{d})$.
    Bounded multiplication operators have as their spectrum
    the essential range, which in this case is the interval
    $[-4d,0]$.  Thus 
    \begin{equation}
        \sigma(-\Delta)
        = [0,4d].
    \end{equation}
    The Laplacian does not have eigenvectors (although the
    plane waves can be seen as ``generalized''
    eigenvectors).  In fact, with a bit of work, we can show
    that $\mu_\psi = \mu_\psi^{\ac}$ for all $\psi \in
    \ell^2(\Z^{d})$. This implies that the whole spectrum is
    essential:
    \begin{equation}
        \sigma(-\Delta)
        = \sigma^{\ac}(-\Delta)
        = \sigma_{\text{ess}}(-\Delta)
        = [0,4d].
    \end{equation}
    
    Now consider a multiplication operator on
    $\ell^2(\Z^{d})$ defined by some real-valued function $V
    \in L^{\infty}(\Z^{d})$. The spectral measure only has
    pure point part, given by a sum of Dirac measures
    \begin{equation}
        \mu_\psi
        = \mu_\psi^{\pp}
        = \sum_{x \in \Z^{d}}^{} |\psi(x)|^2 \delta_{V(x)}.
    \end{equation}
    Its eigenvectors consist of the localized functions
    $\delta_x \in \ell^2(\Z^{d})$ with corresponding
    eigenvalues $V(x)$ with $x \in \Z^{d}$. The closure of
    the eigenvalues gives the pure point spectrum. We
    require more information on the function $V$ to
    determine the nature of the discrete and essential
    spectrum. By definition we know that the essential
    spectrum will consist of all infinitely degenerate
    eigenvalues $V(x)$ along with the accumulation points of
    the full set of eigenvalues.

    \subsection{Anderson localization}

    Now consider the Schrödinger operator $H = -\Delta + V$,
    where $V$ is a multiplication operator whose action is
    to multiply a function $\psi$ by an IID random variable
    with density $\nu$ at each point of $\Z^{d}$.  To be
    very precise requires defining the appropriate measure
    spaces but we will ignore this, and only note that with
    a little ergodic theory one can show that
    \begin{equation}
        \sigma(H)
        = [0,4d] + \text{supp } \nu,
    \end{equation}
    with probability one. The model described by the random
    Schrödinger operator $H$ is well-known as the
    \textit{Anderson tight-binding model} on the ``lattice''
    $\Z^{d}$. It describes a simplified picture (no
    particle interactions) of the transport of an electron
    in a disordered medium, e.g. in an impure crystal.
    Originally introduced by P. W.  Anderson in 1958, it has
    stimulated intense research in both solid state physics
    and more recently (through generalizations of course) to
    condensed matter theory.  The physicists have ``known''
    much about this model for a long time but somewhat
    surprisingly, there are still many outstanding
    mathematical conjectures. The main phenomenon studied in
    these models is that of \textit{localization}. Roughly
    speaking, in a periodic crystal, one expects electronic
    conductance, but the inclusion of disorder through these
    random potentials can create an effect due to
    interference (also only conjectured), in which the
    eigenfunctions of $H$ become exponentially localized in
    the lattice. A common consequence of this is that
    diffusion (as measured by the spreading of the wave
    function) under the time evolution $e^{-it H}$ is
    suppressed and conductance no longer occurs.

    For the one dimensional lattice, $H$ has pure point
    spectrum with exponential localization occurring for all
    strengths of disorder. For $d \ge 2$, pure point spectra
    with exponential eigenfunctions have been proven to
    exist at the ``band edges'', i.e., at the extremes of
    the spectrum, and an entirely pure point spectrum holds
    for sufficiently strong disorder. It is conjectured that
    for $d = 2$ there is no absolutely continuous part, but
    proof of the emptiness of $\sigma^{\ac}(H)$ is still
    wanting! Note that for the one-dimensional lattice,
    $\sigma(H) = [0,4] + \text{supp } \nu$, and since
    $\sigma(H) = \sigma^{\pp}(H)$, its eigenvalues form a
    dense subset of the spectrum, but no ``extended states''
    exist. Physical theory (like the
    \href{https://journals.aps.org/prl/abstract/10.1103/PhysRevLett.42.673}{\textit{scaling
    theory}}) and substantial numerical evidence show that a
    ``mobility edge'', signalling a phase transition from an
    insulator to a metal, exists for $d \geq 3$.

    Things get much more complicated when considering
    interactions and relatively recent results indicating
    that (many-body) localization is an example of a
    failure of the
    \href{https://en.wikipedia.org/wiki/Eigenstate_thermalization_hypothesis}{\textit{eigenstate thermalization
    hypothesis}} have restored an intense interest in the
    field. As of now rigorous results exist only for the
    simplest of spin chains. Furthermore, the phenomenon of
    localization has ties to many other areas of research,
    including percolation theory and most interestingly, to
    the field of quantum chaology!

\end{document}

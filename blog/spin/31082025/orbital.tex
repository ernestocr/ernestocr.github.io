\documentclass[a4paper]{article}

\usepackage[margin=1.5in]{geometry}
\usepackage[utf8]{inputenc}
\usepackage[T1]{fontenc}
\usepackage{textcomp}
\usepackage{amsmath}
\usepackage{amssymb}
\usepackage{amsthm}
\usepackage{mathtools}
\usepackage{bm}
\usepackage{mathrsfs}
\usepackage{hyperref}

%\usepackage{newtxtext,newtxmath}
%\usepackage{tgtermes}
%\usepackage[italic]{mathastext}

%\usepackage{enumitem}
%\usepackage{hyperref}
%\usepackage{graphicx}
%\usepackage{quantikz2}

\DeclareMathOperator{\Tr}{Tr}
\DeclareMathOperator{\spn}{span}
\DeclareMathOperator{\ran}{ran}

\DeclareMathOperator{\N}{\mathbf{N}}
\DeclareMathOperator{\Z}{\mathbf{Z}}
\DeclareMathOperator{\Q}{\mathbf{Q}}
\DeclareMathOperator{\R}{\mathbf{R}}
\DeclareMathOperator{\C}{\mathbf{C}}
\DeclareMathOperator{\F}{\mathbf{F}}

\DeclareMathOperator{\GL}{\mathsf{GL}}
\DeclareMathOperator{\gl}{\mathfrak{gl}}
\DeclareMathOperator{\SO}{\mathsf{SO}}
\DeclareMathOperator{\SU}{\mathsf{SU}}
\DeclareMathOperator{\so}{\mathfrak{so}}
\DeclareMathOperator{\su}{\mathfrak{su}}

%\DeclarePairedDelimiter\bra{\langle}{\rvert}
%\DeclarePairedDelimiter\ket{\lvert}{\rangle}
%\DeclarePairedDelimiterX\braket[2]{\langle}{
%\rangle}{#1\,\delimsize\vert\,\mathopen{}#2}

%\theoremstyle{definition}
%\newtheorem{defn}{Definition}
%\newtheorem*{sol}{Solution}
%\theoremstyle{plain}
%\newtheorem{thm}{Theorem}
%\newtheorem{lem}{Lemma}
%\newtheorem{cor}{Corollary}
%\newtheorem{exa}{Example}
%\newtheorem{exeinner}{Exercise}
%\newenvironment{exe}[1]{%
%    \IfBlankTF{#1}
%    {\renewcommand{theexeinner}{\unskip}}
%    {\renewcommand\theexeinner{#1}}%
%    \exeinner
%}{\endexeinner}

\title{Orbital angular momentum, Lie groups and Lie
algebras}
\author{Ernesto Camacho}
\begin{document}
    \maketitle

    I've been trying to understand angular momentum in
    quantum systems and thought it would be a good idea to
    write some notes here. My goal is to give a basic
    introduction to the theory in sufficient rigor without
    ignoring some of the physical consequences of the
    mathematics. This will be spread out in multiple posts.
    In this brief post, I'll give the basic setup of the
    theory of orbital angular momentum, and how the
    resulting operators directly relate to both $\SO(3)$ and
    $\so(3)$.  Doing so, we'll end up where the physics
    literature usually starts.

    Before doing so, here is a small rant. Symmetries play a
    fundamental role in physics for reasons that, although
    very natural, are still quite mysterious to me. In
    particular, certain Lie groups play a central role in
    both classical and quantum mechanics. In classical
    mechanics, the angular momentum components can be
    thought of as the Hamiltonian generators of rotations.
    This interpretation carries over quite nicely to quantum
    systems via Stone's theorem. In the case of a particle
    moving in $\R^3$ (ignoring spin), the rotation group
    $\SO(3)$ can be represented by a strongly-continuous
    unitary representation, and its infinitesimal
    generators, the well known orbital angular momentum
    operators, can be obtained from a representation of its
    Lie algebra $\so(3)$. A natural question that has been
    bugging me for a while arises here:
    \begin{quote}
        From a physical point of view, which is more
        fundamental: the Lie group representing the physical
        symmetries, or its Lie algebra?
    \end{quote}
    After all, aren't the angular momentum operators
    \textit{the observables} of the system? Isn't also true
    that the physics literature usually starts
    with the angular momentum operators and their
    commutation relations, extracting as much information as
    possible to solve problems of interest, like those
    dealing with rotational invariance? In such cases the
    operators representing $\so(3)$ also carry physical
    meaning as \textit{constants of motion}.  Moreover, to
    describe \textit{all} spin systems, don't we need to
    identify the irreducible representations of $\so(3)$ or
    equivalently, those of the Lie group $\SU(2)$? I believe
    these are valid arguments (and there are other more
    intriguing, see this
    \href{https://physics.stackexchange.com/questions/492965/why-lie-algebras-if-what-we-care-about-in-physics-are-groups}{physics.stackexchange}
    post), and I can't disagree entirely yet. However, I
    believe we should not lose sight of the fact that it is
    the \textbf{group structure} that fundamentally encodes
    the physical symmetry, not its linearization.  In the
    case at hand, $\SO(3)$ is the group of spatial
    rotations, not its Lie algebra and not $\SU(2)$.
    Regarding spin, the confusion seems to stem from a
    failure of acknowledging the fundamentally
    \textit{projective} nature of the quantum state space.
    The theorems of Wigner and Kadison tell us that the
    representations of the symmetry groups are at best
    projective. For example, all spin representations arise
    from $\SO(3)$ but only when we distinguish between
    unitary and projective unitary representations.  The
    half-integer spin representations appear as the latter.
    The introduction of $\SU(2)$ to fully described quantum
    angular momentum is done mainly for technical reasons,
    as the projective representations of $\SO(3)$ can be
    lifted to true unitary representations at the level of
    its universal covering group, and unitary
    representations are usually easier to handle.
    Furthermore, from a mathematical point view, we must not
    forget that in general a Lie group will contain global
    information that is lost when passing to its Lie
    algebra. This surely has physical implications.

    Enough of that, let's begin. The system we will consider
    is that of a particle moving in $\R^3$, ignoring spin
    for now. The pure quantum states can be represented by
    elements of the Hilbert space $\mathcal{H} =
    L^2(\R^3,dx)$. The group of spatial rotations in $\R^3$
    is $\SO(3)$, the compact and connected Lie group of
    orthogonal $3 \times 3$ matrices with unit determinant.
    On $\mathcal{H}$, the group carries a natural
    representation $\Pi : \SO(3) \to
    \mathcal{B}(\mathcal{H})$ defined as
    \begin{equation}
        \label{eqn:SO(3)-scur}
        (\Pi(g) \psi)(x)
        :=
        \psi(g^{-1} x),
        \quad
        \psi \in \mathcal{H},
    \end{equation}
    for all $g \in \SO(3)$. The representation $\Pi$ is
    unitary and strongly-continuous.  We wish to associate
    $\Pi$ with representations of the Lie algebra $\so(3)$,
    in terms of skew-self-adjoint operators on
    $\mathcal{H}$.  As a reminder, the Lie algebra
    $\mathfrak{g}$ of a Lie group $G$ is defined to be the
    tangent space at the identity element $e \in G$. In the
    case of the rotation group, its Lie algebra consists of
    the traceless skew-symmetric $3 \times 3$ real matrices.
    Any element $A \in \so(3)$ defines the one-parameter
    subgroup $\R \ni t \mapsto \exp(t A) \in \SO(3)$. Using
    $\Pi$, we can define a strongly-continuous unitary
    one-paramter group 
    \begin{equation}
        \label{eqn:scuopg}
        \R \ni t
        \mapsto
        \Pi(\exp(t A))
        \in \mathcal{B}(\mathcal{H}), \quad \forall A \in
        \so(3).
    \end{equation}
    By Stone's theorem there exists a unique self-adjoint
    operator $\hat{A} : D(\hat{A}) \to \mathcal{H}$ that
    satisfies
    \begin{equation}
        \Pi(\exp(tA))
        = \exp\left(-it \hat{A}\right),
        \quad
        \forall t \in \R.
    \end{equation}
    We call $\hat{A}$ the self-adjoint \textit{generator}
    (of the strongly-continuous unitary one parameter group)
    associated to $A$. Its domain is given by
    \begin{equation}
        D(\hat{A})
        =
        \left\{
            \psi \in \mathcal{H}
            \, : \,
            \lim_{t \to 0} \frac{1}{t}(\Pi(\exp(tA)) -
            I)\psi
            \;
            \text{exists}
        \right\},
    \end{equation}
    and its image on $\mathcal{H}$ can be computed as
    \begin{equation}
        \hat{A}\psi
        =
        i \frac{d}{dt}\bigg|_{t=0} 
        \Pi(\exp(tA)) \psi
        = 
        \lim_{t \to 0} \frac{i}{t}(\Pi(\exp(tA)) - I)\psi,
        \quad
        \forall \psi \in D(\hat{A}).
    \end{equation}
    Actually computing the Stone domain is usually
    non-trivial, moreover, nothing guarantees that the Stone
    domains of any two generators will coincide. So if we
    have any hope of defining a representation of the Lie
    algebra, we will need to find a suitable common
    invariant dense domain for all generators. For the case
    at hand, the Schwartz space of rapidly decreasing test
    functions $\mathscr{S}(\R^3)$ works just fine. But,
    there is an alternative domain that actually holds for
    arbitrary strongly-continuous unitary representations
    $\Pi$ of a (finite dimensional) real Lie group $G$, the
    so-called
    \href{https://en.wikipedia.org/wiki/G%C3%A5rding_domain}{Gårding
    domain}.
    %To define this subspace, let $f \in C^{\infty}_0(G)$ be
    %a smooth real-valued function on $G$ with compact
    %support and $x \in \mathcal{H}$. Define the vector $x[f]
    %:= \int_G f(g) U_g x \, dg$, where $dg$ is the
    %left-invariant Haar measure on $G$ and the integral is
    %defined in the weak sense, i.e., $x[f]$ is the unique
    %vector such that
    %\begin{equation}
    %    \langle y, x[f] \rangle
    %    =
    %    \int_G f(g) \langle u, U_g x \rangle \, dg
    %    \quad
    %    \forall y \in \mathcal{H}.
    %\end{equation}
    %The Gårding space $D_G^{(U)}$ of $U$ and $G$ is the
    %finite span of the $x[f] \in \mathcal{H}$ with $f \in
    %C_c^{\infty}(G)$ and $x \in \mathcal{H}$.
    This domain, denoted by $D_G^{(\Pi)}$ can be defined as
    the subspace of all $\psi \in \mathcal{H}$ such that the
    map $g \mapsto \Pi(g) \psi$ is an element of
    $C^{\infty}(G)$. There are many properties of the
    Gårding space that make it very useful for dealing with
    representations of Lie groups and algebras, e.g., there
    is a natural correspondance between the action of the
    unitary representation, and the usual left action of the
    group on its space of functions. For our
    purposes only the following few properties are important
    (for more, please see Valter Moretti's
    \textit{\href{https://link.springer.com/book/10.1007/978-3-030-18346-2}{Fundamental
    Mathematical Structures of Quantum Mechanics}}, from
    which this post takes plenty).
    First, $D_G^{(\Pi)}$ is dense in $\mathcal{H}$.
    Second, it is invariant under the representation,
    i.e., $\Pi(g) D_G^{(\Pi)} \subset D_G^{(\Pi)}$ for
    all $g \in G$.  Third, the map $\pi : \mathfrak{g}
    \to \mathcal{L}(\mathcal{H})$ defined as
    \begin{equation}
        \pi(A)
        =
        -i \hat{A} \big|_{D_G^{(\Pi)}},
    \end{equation}
    takes the element $A \in \mathfrak{g}$ to a
    skew-symmetric operator defined by restricting the Stone
    generator associated to $A$ to the Gårding domain. With
    some work, it can be shown that $\pi$ is an $\R$-linear
    representation of the Lie algebra $\mathfrak{g}$, i.e.,
    $\pi(A)D_{G}^{(\Pi)} \subset D_{G}^{(\Pi)}$ and
    \begin{equation}
        [\pi(A), \pi(A')]
        =
        \pi([A,A]),
        \quad
        \forall A,A' \in \mathfrak{g}.
    \end{equation}
    The final important property is that the closure of the
    representation is equal to the Stone generator, i.e.,
    \begin{equation}
        \hat{A}
        =
        \overline{i\pi(A)},
        \quad
        \forall A \in \mathfrak{g}.
    \end{equation}
    Setting $G = \SO(3)$ and $\mathfrak{g}
    = \so(3)$, we now see that from a unitary
    representation of the rotation group, we've obtained a
    representation of the associated Lie algebra, by
    restricting the generators of the unitary representation
    and multiplying by a factor in order to preserve the
    bracket. Let us now work out what $\pi : \so(3) \to
    \mathcal{L}(\mathcal{H})$ looks like explicitly. A basis
    for $\so(3)$ is given by the traceless and
    skew-symmetric matrices
    \begin{equation}
        \label{eqn:so(3)-basis}
        F_1
        =
        \begin{pmatrix} 
            0 & 0 & 0 \\
            0 & 0 & -1 \\
            0 & 1 & 0 
        \end{pmatrix},
        \quad
        F_2
        =
        \begin{pmatrix} 
            0 & 0 & 1 \\
            0 & 0 & 0 \\
            -1 & 0 & 0
        \end{pmatrix}
        \quad\text{and}\quad
        F_3
        =
        \begin{pmatrix} 
            0 & -1 & 0 \\
            1 & 0 & 0 \\
            0 & 0 & 0
        \end{pmatrix}.
    \end{equation}
    These satisfy the well known commutation relations 
    \begin{equation}
        \label{eqn:so(3)-comm}
        [F_i, F_j]
        =
        \sum_{k}^{} \varepsilon_{ijk} F_k.
    \end{equation}
    Take $F_3$ as an example. The $\SO(3)$ subgroup
    generated by $F_3$ can be expressed as
    \begin{align}
        \exp(t F_3)
        =
        I + \sin(t) F_3 + (1-\cos(t)) F_3^2
        = 
        \begin{pmatrix} 
            \cos t & -\sin t & 0 \\
            \sin t & \cos t & 0 \\
            0 & 0 & 1
        \end{pmatrix}, 
    \end{align}
    which is of course nothing more than the matrix
    corresponding to a rotation of angle $t$ about the
    $x_3$-axis of some coordinate frame. Using $\Pi$, the
    generator  corresponding to $F_3$ (which we now denote
    by $L_3 := \hat{F_3}$) has the following effect on any
    $\psi \in D(L_3)$:
    \begin{align}
        (L_3\psi)(\bm x)
        &= i \frac{d}{dt}\bigg|_{t=0}\Pi(\exp(t
        F_3))\psi(\bm x) \\
        &= i \frac{d}{dt}\bigg|_{t=0} \psi(\exp(-t F_3) \bm
        x) \\
        &= i \frac{d}{dt}\bigg|_{t=0} \psi(\cos(t) x_1 +
        \sin(t) x_2, -\sin(t) x_1 + \cos(t) x_2, x_3) \\
        &= i \left(
            \partial_1 \psi(\bm x) x_2 - \partial_2
            \psi(\bm x) x_1
        \right) \\
        &= (X_1 P_2 - X_2 P_1)\psi(\bm x) 
        \quad
        \forall \psi \in D(L_3),
    \end{align}
    where $X_i$ and $P_j$ are the self-adjoint position and
    momentum operators restricted to the Stone domain of
    $L_3$.  The self-adjoint operator $L_3$ is the
    \textit{orbital angular momentum} operator for the
    $x_3$-axis. This result matches what we obtain by the
    usual quantization of the classical angular momentum
    vector $\vec L = \vec x \wedge \vec p$. Repeating the
    procedure for the other two orthogonal axis we obtain
    \begin{equation}
        L_k
        =
        \sum_{i,j}^{} \varepsilon_{kij} X_i P_j.
    \end{equation}
    It seems so little has been gained by taking the domain
    issues of unbounded operators seriously, but what do I
    know. On the Gårding domain $D_{\SO(3)}^{(\Pi)}$, a quick
    calculation reveals that the $L_k$ satisfy the
    commutation relations
    \begin{equation}
        [L_i, L_j]
        = i \sum_{k}^{} \varepsilon_{ijk} L_k.
    \end{equation}
    It immediately follows that the $\pi(F_k)$ satisfy
    \begin{align}
        [\pi(F_i), \pi(F_j)]
        = [-i L_i, -i L_j]
        = -i \sum_{k}^{} \varepsilon_{ijk} L_k
        = \sum_{k}^{} \varepsilon_{ijk} \pi(F_k).
    \end{align}
    Thus $\pi$ preserves the Lie bracket of
    $\so(3)$, as it should. So restricting to the
    Gårding domain, the real span of the $\pi(F_k)$
    constitute a Lie algebra representation of $\so(3)$ in
    terms of skew-symmetric operators on $\mathcal{H}$. Just
    as we desired!

    So we've briefly shown how the orbital angular
    momentum operators of quantum mechanics are the
    infinitesimal generators of a strongly-continuous
    unitary representation of the rotation group, and at the
    same time, how they \textit{induce} a representation of
    the associated Lie algebra. Usually the next task
    consists of identifying the finite unitary irreducible
    representations of $\SO(3)$. This is conveniently done
    by identifying the finite unitary irreducible
    representations of $\so(3)$ or equivalently those of
    $\SU(2)$.  Doing so, we will encounter the fact that
    only some of these representations can actually be
    ``found inside'' the infinite-dimensional representation
    of the rotation group. In the physics jargon, one says
    that only the integer representations appear as
    subrepresentations of the orbital angular momentum.  Of
    course, what is actually meant is that \textit{truely
    unitary} representations only exist for integer spin.
    In the next post I hope to show how we can realize these
    integer spin representations in $\mathcal{H}$ and how
    they're used for solving the Schrödinger equation for
    rotationally invariant Hamiltonians. 
    
    


\end{document}

\documentclass[a4paper]{article}

\usepackage[margin=1.5in]{geometry}
\usepackage[utf8]{inputenc}
\usepackage[T1]{fontenc}
\usepackage{textcomp}
\usepackage{amsmath}
\usepackage{amssymb}
\usepackage{amsthm}
\usepackage{mathtools}
\usepackage{bm}
\usepackage{tikz}
\usepackage{tikz-3dplot}

%\usepackage{newtxtext,newtxmath}
%\usepackage{tgtermes}
%\usepackage[italic]{mathastext}

%\usepackage{enumitem}
%\usepackage{hyperref}
%\usepackage{graphicx}
%\usepackage{quantikz2}

\DeclareMathOperator{\N}{\mathbf{N}}
\DeclareMathOperator{\Z}{\mathbf{Z}}
\DeclareMathOperator{\Q}{\mathbf{Q}}
\DeclareMathOperator{\R}{\mathbf{R}}
\DeclareMathOperator{\C}{\mathbf{C}}
\DeclareMathOperator{\F}{\mathbf{F}}

%\DeclarePairedDelimiter\bra{\langle}{\rvert}
%\DeclarePairedDelimiter\ket{\lvert}{\rangle}
%\DeclarePairedDelimiterX\braket[2]{\langle}{
%\rangle}{#1\,\delimsize\vert\,\mathopen{}#2}

%\theoremstyle{definition}
%\newtheorem{defn}{Definition}
%\newtheorem*{sol}{Solution}
%\theoremstyle{plain}
%\newtheorem{thm}{Theorem}
%\newtheorem{lem}{Lemma}
%\newtheorem{cor}{Corollary}
%\newtheorem{exa}{Example}
%\newtheorem{exeinner}{Exercise}
%\newenvironment{exe}[1]{%
%    \IfBlankTF{#1}
%    {\renewcommand{theexeinner}{\unskip}}
%    {\renewcommand\theexeinner{#1}}%
%    \exeinner
%}{\endexeinner}

\title{John Bell's HVM for a single spin-1/2 particle}
\author{Ernesto Camacho}
\begin{document}
    \maketitle

    In John Bell's original 1964 paper \textit{On the
    Einstein Podolsky Rosen Paradox} he gives an
    illustration of a hidden variable model that reproduces
    all of the spin measurement statistics of quantum theory
    for a single particle of spin-1/2. It was a common held
    belief at the time that such models cannot exist, and so
    his illustration served as motivation for his subsequent
    analysis. He quickly presents how the model is defined
    and skips a lot of the calculations, and so for my
    understanding, I've filled those in. 

    He starts out with a spin-1/2 particle in a pure state
    with polarization given by some vector $\vec p$. The
    hidden variable that he has chosen for this model is
    given by a unit vector $\vec \lambda$ that is
    distributed uniformly over the hemisphere $\vec \lambda
    \cdot \vec p > 0$:

    \begin{figure}[ht]
        \centering
        \tdplotsetmaincoords{60}{110}

        %define polar coordinates for some vector
        \pgfmathsetmacro{\radius}{1}
        \pgfmathsetmacro{\thetavec}{0}
        \pgfmathsetmacro{\phivec}{0}


        %start tikz picture, and use the tdplot_main_coords style to implement the display 
        %coordinate transformation provided by 3dplot
        \begin{tikzpicture}[scale=3,tdplot_main_coords]
            %draw the main coordinate system axes
            %\draw[thick,->] (0,0,0) -- (-1,0,0) node[anchor=south]{$z$};
            %\draw[thick,->] (0,0,0) -- (0,1,0) node[anchor=north west]{$x$};
            %\draw[thick,->] (0,0,0) -- (0,0,1) node[anchor=south]{$y$};
            \tdplotsetcoord{P}{1}{0}{90}
            \tdplotsetcoord{L}{1}{45}{90}

            \draw[thick,->] (0,0,0) -- (P) node[anchor=south west]{$\vec p$};
            \draw[thick,->,blue] (0,0,0) -- (L)
                node[anchor=south west]{$\vec \lambda$};

            \tdplotsetthetaplanecoords{\phivec}

            %draw some dashed arcs, demonstrating direct arc drawing
            %\draw[dashed,tdplot_rotated_coords] (\radius,0,0) arc (0:90:\radius);

            \draw[dashed] (\radius,0,0) arc (0:360:\radius);
            \shade[ball color=blue!10!white,opacity=0.2] (1cm,0) arc (0:-180:1cm and 5mm) arc (180:0:1cm and 1cm);
            % (-z x y)
            %\draw (0, 1, 0) node [circle, fill=blue, inner sep=.02cm] () {};
            %\draw (0, 0, 1) node [circle, fill=green, inner sep=.02cm] () {};
            %\draw (-1, 0, 0) node [circle, fill=red, inner sep=.02cm] () {};
        \end{tikzpicture}
        \caption{The hemisphere defined by $\vec \lambda
        \cdot \vec p > 0$.}
        \label{fig:hemi}
    \end{figure}

    Let $S^2$ be the unit sphere which we will use as the
    set of spin measurement directions. Let $\Lambda$ be the
    set of unit vectors satisfying the hemisphere condition.
    The model is a function $A : \Lambda \times S^2 \to
    \{-1,1\}$ defined as
    \begin{equation}
        A(\vec \lambda, \vec a)
        = \text{sign } \vec \lambda \cdot \vec a',
    \end{equation}
    where $\vec a'$ is a unit vector that depends on the
    polarization $\vec p$ and the measurement direction
    $\vec a$. This angle is chosen in such a way that the
    statistics of the measurements are reproduced. This is
    done by calculating the expectation value of $A$ over
    the hidden variable given by the uniform density
    $\rho(\lambda)$. John does not state any details here
    but we can assume that he is using the Lebesgue measure
    $d\Omega(\vec \lambda) = \sin\theta \, d\theta d\phi$ on
    $S^2$ properly normalized over the hemisphere of
    interest. Since $\int_{\Lambda} \, d\Omega(\lambda) =
    2\pi$, the distribution is 
    \begin{equation}
        \rho(\vec \lambda)
        =
        \begin{cases}
            \frac{1}{2\pi} & \text{if } \vec \lambda \cdot
            \vec p > 0 \\
            0 & \text{elsewhere.}
        \end{cases}
    \end{equation}
    John then states that the expected value of $A$ given
    the distribution $\rho$ is equal to $1 - 2\theta' / \pi$ 
    where $\theta'$ is the angle between the soon to be
    determined vector $\vec a'$ and $\vec p$. We now look
    at the details of this calculation. By definition, the
    expectation value is
    \begin{align}
        \int_{\Lambda}
        A(\vec \lambda, \vec a) \rho(\vec \lambda) \,
        d\Omega(\vec \lambda)
        &= \int_{\Lambda}
        (\text{sign } \vec \lambda \cdot \vec a') \rho(\vec
        \lambda) \, d\Omega(\vec \lambda) \\
        &= \int_{\Lambda \cap (\vec \lambda \cdot \vec a' >
        0)} \rho(\vec \lambda) \, d\Omega(\vec \lambda)
        - \int_{\Lambda \cap (\vec \lambda \cdot \vec a' <
        0)} \rho(\vec \lambda) \, d\Omega(\vec \lambda).
    \end{align}
    To evaluate the integrals we must compute the area of
    the intersection of two hemispheres. Let $\gamma$ be the
    angle between two vectors $\vec a$ and $\vec b$, which
    define the hemispheres $H_1$ and $H_2$ respectively.  We
    orient the coordinate system so that both vectors lie on
    the $xy$-plane with $\vec a$ parallel to the $x$-axis
    and $\gamma \in [0,2\pi]$.  Symmetry dictates that the
    area will be the same for $\gamma$ and $\gamma + \pi$,
    so we only consider $\gamma \in [0,\pi]$. Under this
    choice, $\vec a$ has coordinates $\theta =
    \frac{\pi}{2}$ and $\phi = 0$ and $\vec b$ has
    coordinates $\theta = \frac{\pi}{2}$ and $\phi =
    \gamma$. Then the area of the intersection of $H_1$ and
    $H_2$ is
    \begin{equation}
        \int_{H_1 \cap H_2} \, d\Omega(\vec \lambda)
        = 2\pi - \int_{0}^{\gamma}\int_{0}^{\pi}
        \sin\theta \, d\theta d\phi
        = 2\pi - 2 \int_{0}^{\gamma} \, d\phi
        = 2\pi - 2\gamma.
    \end{equation}
    Going back to the calculation of the expectation value,
    we recall that $\Lambda$ is the hemisphere defined by
    $\vec p$, while $\vec a'$ defines the other hemisphere,
    therefore
    \begin{equation}
        \int_{\Lambda \cap (\vec \lambda \cdot \vec a' > 0)}
        \rho(\vec \lambda) \, d\Omega(\vec \lambda)
        = \frac{1}{2\pi} \int_{\Lambda \cap (\vec \lambda
        \cdot \vec a' > 0)} \,
        d\Omega(\vec \lambda)
        = \frac{1}{2\pi} \left( 2\pi - 2\theta' \right) 
        = 1 - \frac{\theta'}{\pi},
    \end{equation}
    where $\theta'$ is the angle between $\vec p$ and $\vec
    a'$. Similarly, the condition $\vec \lambda \cdot \vec a
    < 0$ defines a hemisphere opposite to the one defined by
    $\vec \lambda \cdot \vec a > 0$. Thus, since $\rho$ is
    normalized we obtain
    \begin{equation}
        \int_{\Lambda \cap (\vec \lambda \cdot \vec a' < 0)}
        \rho(\vec \lambda) \, d\Omega(\vec \lambda)
        = 1 - \left(1 - \frac{\theta'}{\pi}\right)
        = \frac{\theta'}{\pi}.
    \end{equation}
    Therefore the expectation value is
    \begin{equation}
        \int_{\Lambda} A(\vec \lambda, \vec a) \rho(\vec
        \lambda) \, d\Omega(\vec \lambda)
        = \left(1 - \frac{\theta'}{\pi}\right) -
        \frac{\theta'}{\pi} 
        = 1 - \frac{2\theta'}{\pi},
    \end{equation}
    which is what John stated in Eqn. (5) of his paper. The
    question now is, how do we define $\vec a'$ so that this
    model reproduces the spin statistics. This is of course
    trivial since the expectation value of a spin
    measurement in direction $\vec a$ for a polarization
    $\vec p$ is 
    \begin{equation}
        \langle \vec a \cdot \vec \sigma \rangle
        = \vec a \cdot \vec p
        = \cos\theta,
    \end{equation}
    where $\theta$ is the angle between $\vec a$ and $\vec
    p$. So \textit{given} the measurement direction and the
    state, the angle $\theta'$ between  $\vec p$ and $\vec
    a'$ must satisfy
    \begin{equation}
        1 - \frac{2\theta'}{\pi}
        = \cos\theta,
    \end{equation}
    which leads us to,
    \begin{equation}
        \theta'
        = \frac{\pi}{2}\left(1 - \cos\theta\right).
    \end{equation}
    So by construction, the expectation value of the model
    $A$, for a given measurement direction $\vec a$, with
    respect to the uniform distribution $\rho$ over the
    hidden variables $\Lambda$, determined by the
    polarization $\vec p$ (a bit of a mouthful), matches
    the expectation value given by quantum theory. In other
    words, the model reproduces all of the spin statistics
    in a deterministic manner, i.e., \textit{if} we knew the
    value of $\vec \lambda$ for some measurement instance,
    then we could predict with certainty the outcome $A(\vec
    \lambda, \vec a)$.  The statistical features of quantum
    theory then only arise because we don't know the value
    of this variable for individual instances. John then
    goes on to show that when considering a bi-partite
    system, under certain assumptions (which are reasonable
    but can be questioned), any hidden variable model is
    constraind by some inequality determined by the
    correlations of the value assignments, yet the
    statistical outcomes determined by quantum theory
    violate this inequality. Such violations have been
    verified experimentally, yet one must be careful not to
    draw over-reaching conclusions about non-locality and
    realism, as a lot comes down to the accepted
    assumptions. I will post more on this later.

        


\end{document}

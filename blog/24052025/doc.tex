\documentclass[a4paper]{article}

\usepackage[margin=1.5in]{geometry}
\usepackage[utf8]{inputenc}
\usepackage[T1]{fontenc}
\usepackage{textcomp}
\usepackage{amsmath}
\usepackage{amssymb}
\usepackage{amsthm}
\usepackage{mathtools}
\usepackage{bm}

%\usepackage{newtxtext,newtxmath}
%\usepackage{tgtermes}
%\usepackage[italic]{mathastext}

%\usepackage{enumitem}
%\usepackage{hyperref}
%\usepackage{graphicx}
%\usepackage{quantikz2}

\DeclareMathOperator{\N}{\mathbf{N}}
\DeclareMathOperator{\Z}{\mathbf{Z}}
\DeclareMathOperator{\Q}{\mathbf{Q}}
\DeclareMathOperator{\R}{\mathbf{R}}
\DeclareMathOperator{\C}{\mathbf{C}}
\DeclareMathOperator{\F}{\mathbf{F}}

%\DeclarePairedDelimiter\bra{\langle}{\rvert}
%\DeclarePairedDelimiter\ket{\lvert}{\rangle}
%\DeclarePairedDelimiterX\braket[2]{\langle}{
%\rangle}{#1\,\delimsize\vert\,\mathopen{}#2}

\theoremstyle{definition}
\newtheorem{defn}{Definition}
\newtheorem*{sol}{Solution}
\theoremstyle{plain}
\newtheorem{thm}{Theorem}
\newtheorem{lem}{Lemma}
\newtheorem{cor}{Corollary}
\newtheorem{exa}{Example}
%\newtheorem{exeinner}{Exercise}
%\newenvironment{exe}[1]{%
%    \IfBlankTF{#1}
%    {\renewcommand{theexeinner}{\unskip}}
%    {\renewcommand\theexeinner{#1}}%
%    \exeinner
%}{\endexeinner}

\title{On the PBR Theorem}
\author{Ernesto Camacho}
\begin{document}
    \maketitle

    It may be surprising that a century later after its
    introduction, experts still do not agree on what the
    quantum state really represents. Should each wave
    function be associated to a physical state of a real
    system or is it no more than a calculational tool
    representing our state of knowledge of said system?
    These questions are themselves an oversimplification,
    and both viewpoints cover up shortcomings but are not
    free of their own sets of problems. As a mathematical
    tool, I feel adopting the statistical interpretation of
    quantum theory to be the most pleasent and reasonable,
    but it is very difficult to explain certain
    experimentally verified results with this mindset. I'm
    of course refering to a number of so-called no-go
    theorems that severly limit the existence of ontological
    models that reproduce the statistics of quantum theory.
    
    Today I want to look at a relatively recent no-go
    theorem that starts out by assuming that the wave
    function represents nothing more than information of a
    subyacent physical state. In 2012 Matthew F. Pusey,
    Jonathan Barrett and Terry Rudolph published a short
    paper titled \textit{On the reality of the quantum
    state} [CITE], in which they showed that under certain
    assumptions, this idea is incompatible the predictions
    of quantum theory. This post is an overview of that
    result.

    The authors start from Harrigan and Spekkens notion of
    the idea that a quantum state corresponds directly to
    reality or represents only information.

    One of the main problems about these types of questions
    is the fuzzy definition of reality. The difficulty is
    not small but one must start from somewhere, PBR's
    definition is the following.

    \begin{defn}
        A physical property of a system is a label $L$ for
        which there exists disjoint probability
        distributions of the system's physical state
        $\lambda$.
    \end{defn}

    We assume there is a set of possible physical states
    $\lambda$ in which a system can be. Let $\Lambda$ be a
    measurable space which models this set. A preparation of
    the quantum state $\psi_i$ is assumed to result in a
    $\lambda$ sampled from a probability distribution
    $\mu_i : \Lambda \to [0,1]$. Assume that it is possible
    to prepare $n$ systems independently,\footnote{.} with
    quantum states $\psi_{x_1},\ldots,\psi_{x_n}$ which
    result in the physical states
    $\lambda_1,\ldots,\lambda_n$ distributed according to
    the product distribution

    \begin{equation}
        \mu_{x_1} \cdot \mu_{x_2} \cdots \mu_{x_n}.
    \end{equation}
   
    Additionally assume that a fixed set of
    $\lambda_1,\ldots,\lambda_n$ fixes the probability for
    the outcome $k$ of a measurement according to some
    conditional probability distribution 

    \begin{equation}
        p(k | \lambda_1,\ldots,\lambda_n).
    \end{equation}

    The law of total probability allows us to calculate the
    operational probabilities as
    
    \begin{equation}
        p(k|\Psi(x_1,\ldots,x_n))
        = \int_{\Lambda^{n}} p(k|\lambda_1,\ldots,\lambda_n)
        \mu_{x_1}(\lambda_1) \cdots \mu_{x_n}(\lambda_n) \,
        d\lambda_1 \cdots d\lambda_n.
    \end{equation}

    Now suppose an experiment is carried out in which with
    high confidence it can be ascertained that the
    probability for each measurement outcome is within
    $\varepsilon$ of the predicted quantum probability for
    some small $\varepsilon > 0$. The final result relates
    $\varepsilon$ to the total variation dsitance between
    $\mu_0$ and $\mu_1$ defined by
    
    \begin{equation}
        D(\mu_0,\mu_1)
        = \frac{1}{2} \int_\Lambda |\mu_0(\lambda) -
        \lambda_1(0)| \, d\lambda.
    \end{equation}

    If $D(\mu_0,\mu_1) = 1$ then the probability of
    $\lambda$ being compatible with both preparations is
    zero. 
    
\end{document}
